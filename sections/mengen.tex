\documentclass[../main.tex]{subfiles}
    \usepackage{amsmath}
    \usepackage{amsfonts}
    \usepackage{listings}
    
\begin{document}
\begin{enumerate}
	\item Gegeben sind die folgenden Teilmengen \(A = \{ 1, 3, 5, 7, 9 \}, B = \{2, 4, 6, 8, 10 \} \) und
	      \(D = \{ 5,6,7,8,9,10\} \).

	      Gib die folgenden Mengen an:
	      \begin{enumerate}
		      \item \(A \cup B \)
		      \item \(A \cap B \)
		      \item \(A \setminus B \)
		      \item \(A \setminus D \)
		      \item \(B \setminus D \)
		      \item \(D \setminus A \)
		      \item \(D \setminus B \)
		      \item \(D \setminus(A \cup B) \)
		      \item \(D \setminus(A \cap B) \)
	      \end{enumerate}
	\item Wie viele Elemente enthält die Potenzmenge \( \mathcal{P}(A) \) einer (endlichen)
	      Menge \(A \) mit \( |A| = n \)? Schreibe z.B. alle Teilmengen von \( \{1,2\} \) oder
	      \( \{1,2,3\} \) auf, und versuche eine Regelmäßigkeit zu erkennen.
	      Wie könnte man die Regelmäßigkeit allgemein beweisen?
	      Zeige dass für endliche Mengen stets \( |A| < |\mathcal{P}(A)| \) gilt.
	\item
	\item
	\item
	\item
\end{enumerate}
\end{document}