\documentclass[../main.tex]{subfiles}
    \usepackage{amsmath}
    \usepackage{listings}
    \usepackage{amsfonts}
    
\begin{document}
\begin{enumerate}
	\item Skizziere den Graph der Funktion \( x \mapsto \log_2 x \) für \( x = 2^{-1000} \dots 1000 \).

	      Lösung:
	      \begin{enumerate}
		      \item
	      \end{enumerate}
	\item Bestimme für beliebiges positives \( b \neq 1 \) folgende Werte:
	      \( \log_b 1 \) und \( \log_b b \).

	      Lösung:
	      \begin{enumerate}
		      \item \( \log_b 1 = 0 \)
		      \item \( \log_b b = 1 \)
	      \end{enumerate}
	\item Finde mit den Rechenregeln für Potenzen und Logarithmen eine Rechenregel für \(
	      log_b \sqrt[n]{x}
	      \).

	      Lösung:
	      \begin{enumerate}
		      \item \( \log_b \sqrt[n]{x} = \log_b x^{\frac{1}{n}} = \frac{1}{n} \cdot \log_b x\)
	      \end{enumerate}
	\item Vereinfache folgende Ausdrücke (\( b, c, x \) und \( y \)
	      seien positiv mit \( b,c \neq 1 \)):
	      \begin{align*}
		      b^{x + \log_b y},
		      \Big( \sqrt{b} \Big)^{\log_b x},
		      \log_c \Big( x^{ \frac{1}{\log_c b}} \Big),
	      \end{align*}

	      Lösung:
	      \begin{enumerate}
		      \item \(  b^{x + \log_b y}
		            = b^x \cdot b^{\log_b y}
		            = b^x \cdot y \)
		      \item \( \Big( \sqrt{b} \Big)^{\log_b x}
		            = b^{\frac{ \log_b x }{ 2 }}
		            = b^{\log_b x} \cdot b^{\frac{1}{2}}
		            = x \cdot \sqrt{b} \)
		      \item \( \log_c \Big( x^{ \frac{1}{\log_c b}} \Big)
		            = \frac{1}{\log_c b} \cdot \log_c x
		            = \frac{ \log_c x}{ \log_c b} \)
	      \end{enumerate}
	\item Vereinfache (es sei \( x > y > 0 \))
	      \[  \ln(x^2 - y^2) - \ln(x - y)\]

	      Lösung:
	      \begin{enumerate}
		      \item \( \ln(x^2 - y^2) - \ln(x - y) \)

		            \( = \ln((x + y)(x - y)) - \ln(x - y) \)

		            \( = \ln(x + y) + \ln(x - y) - \ln(x - y)
		            = \ln (x + y) \)
	      \end{enumerate}
	\item Wenn für Abszisse (vulgo ”\( x \)-Achse“) und Ordinate logarithmische
	      Maßstäbe verwendet werden —
	      wie sehen dann die Graphen von Potenzfunktionen \( x \mapsto x^n \) aus?

	      Lösung:
	      \begin{enumerate}
		      \item Die Graphen sind eine Gerade.
	      \end{enumerate}
	\item Jaja, ich weiß schon, dass die allermeisten von Ihnen nicht vorhaben, jemals auf einem Rechenschieber zu rechnen.
	      Zum Logarithmen-Üben ist das Ding (bzw. eine Vorstellung davon) aber immer noch praktisch!
	      Manchmal wollen auch Informatiker die Länge der Diagonale eines Quadrats berechnen — markiere dazu auf dem
	      Informatiker-Rechenschieber den Wert \( \sqrt{2} \).

	      Markiere nun noch denjenigen Wert, mit dem man die Kantenlänge eines Würfels multiplizieren muss,
	      um die Kantenlänge eines Würfels mit doppeltem Volumen zu erhalten.

	      Wie kann man mit unserem Rechenschieber — ohne die Skalen zu verlängern —
	      den Wert von \( 8 \cdot 512 \) ablesen?
	      (Tipp: am echten Rechenschieber heißt diese Technik Durchschieben. Bei unserm Modell ist ggf.
	      mal wieder die Näherung \( 2^{10} \approx 1000 \) hilfreich.)

	      Lösung:
	      \begin{enumerate}
		      \item
	      \end{enumerate}
	\item In wieviel Jahren hat sich eine mit einen Zinssatz von \( p\% \)
	      im Jahr verzinste Geldanlage (incl. Zinseszins) verdoppelt?
	      Berechne diese Dauer für \( p = 1,2,3,4 \) und \( 10 \) und
	      vergleiche die Ergebnisse mit der Faustregel ”\( 70 \) Jahre geteilt durch
	      Zinssatz“.

	      Lösung:
	      \begin{enumerate}
		      \item Die Geldanlage hat nach einer Verzsinung den Wert:

		            \( x_{neu} = x_{alt} + x_{alt} \cdot \frac{p}{100}  \)

		            Nach einer weiteren Verzinsung hat sie den Wert:

		            \(  ( x_{alt} + x_{alt} \cdot \frac{p}{100} ) + ( x_{alt} + x_{alt} \cdot \frac{p}{100} ) \cdot \frac{p}{100}\)

		            \( =  ( x_{alt} + x_{alt} \cdot \frac{p}{100} ) \cdot ( 1+ \frac{p}{100}) \)

		            \( = x_{alt} (1 + \frac{p}{100}) (1 + \frac{p}{100}) \)

		            \( = x_{alt} (1+ \frac{p}{100})^2 \)

		            Diese Formel lässt sich für beliebige Verzinsungen verallgemeinern
		            \[ x_{alt} \Big(1 + \frac{p}{100} \Big)^n \]

		            Nun kann man berechnen, wie lange es dauert, bis sich das Startkapital verdoppelt hat.

		            \( 2 \cdot x_{alt} = x_{alt} \Big(1 + \frac{p}{100} \Big)^n \) \( | \div x_{alt} | \ln() \)

		            \( \ln 2 = n \cdot \ln(1 + \frac{p}{100}) \) \( | \div \ln (1 +\frac{p}{100}) \)

		            \( n = \frac{\ln 2}{\ln (1 + \frac{p}{100})} \)

		            Durch Einsetzen von Werten für \( p \) kann die benötigte Anzahl an Jahren errechnet werden,
		            bis sich das Startkapital verdoppelt hat.

		            In der zweiten Zeile der Tabelle ist zum Vergleich das
		            Ergebnis laut der Faustregel angegeben.
		            \[
			            \begin{array}{c|ccccc}
				            p          & 1     & 2  & 3     & 4     & 10   \\
				            \hline
				            Formel     & 69.66 & 35 & 23.45 & 17.67 & 7.27 \\
				            Faustregel & 70    & 35 & 23.33 & 17.5  & 7    \\
			            \end{array}
		            \]
	      \end{enumerate}
	\item Zeige, dass \( ld 10 \) keine rationale Zahl ist
	      (es gibt keine ganzen Zahlen \( x , y \) mit \( ld 10 = x / y \)).

	      Lösung:
	      \begin{enumerate}
		      \item Es wird ein Widerspruchsbeweis verwedet, bei dem davon ausgegangen wird,
		            dass die zu widerlegende Annahme wahr ist. Falls dann durch logische Schlüsse ein
		            Widerspruch hergeieltet werden kann muss die Annahme falsch sein, wodurch die zu
				beweisende Bedingung bewiesen ist.
				
				Annahme: Es gibt ganze Zahlen \( x,y \in \mathbb{Z} \), für die gilt:
				\( ld 10 = \frac{x}{y} \)

				\( ld 10 = \frac{x}{y} \) \( | 2^x \)

				\( 10 = 2^\frac{x}{y} \) \( | x^y \)

				\( 10^y = 2^x \)

				Dies is umöglich, solange \( x, y \neq 0 \), was ausgeschlossen werden kann, da sonst
				\( ld 10 = 1 \).

				Somit ist \( 10^y \neq 2^x \) für alle ganzen Zahlen \( x,y \). Damit ist die
				Annahme, dass \( ld 10 =  x / y  \) widerlegt, wodurch \( ld 10  \) irrational ist.
	      \end{enumerate}
\end{enumerate}
\end{document}