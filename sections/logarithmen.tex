\documentclass[../main.tex]{subfiles}
    \usepackage{amsmath}
    \usepackage{listings}
    
\begin{document}
\begin{enumerate}
	\item Skizziere den Graph der Funktion \(
	      x \mapsto ld x
	      \) für \(
	      x = 2^{-1000} \dots 1000
	      \).

	      Lösung:
	      \begin{enumerate}
		      \item
	      \end{enumerate}
	\item Bestimme für beliebiges positives \(
	      b \neq 1
	      \) folgende Werte: \(
	      log_b 1
	      \) und \(
	      log_b b
	      \).

	      Lösung:
	      \begin{enumerate}
		      \item
	      \end{enumerate}
	\item Finde mit den Rechenregeln für Potenzen und Logarithmen eine Rechenregel für \(
	      log_b \sqrt[n]{x}
	      \).

	      Lösung:
	      \begin{enumerate}
		      \item
	      \end{enumerate}
	\item Vereinfache folgende Ausdrücke (\(
	      b, c, x
	      \) und \(
	      y
	      \) seien positiv mit \(
	      b,c \neq 1
	      \)):
	      \begin{align*}
		      b^{x + log_b y},
		      \Big( \sqrt{b} \Big)^{log_b x},
		      log_c \Big( x^{ \frac{1}{log_c b}} \Big),
	      \end{align*}

	      Lösung:
	      \begin{enumerate}
		      \item
	      \end{enumerate}
	\item Vereinfache (es sei \(
	      x > y > 0
	      \))
	      \begin{align*}
		      ln(x^2 - y^2) - ln(x - y)
	      \end{align*}

	      Lösung:
	      \begin{enumerate}
		      \item
	      \end{enumerate}
	\item Wenn für Abszisse (vulgo ”\(
	      x
	      \)-Achse“) und Ordinate logarithmische Maßstäbe verwendet werden —
	      wie sehen dann die Graphen von Potenzfunktionen \(
	      x \mapsto x^n
	      \) aus?

	      Lösung:
	      \begin{enumerate}
		      \item
	      \end{enumerate}
	\item Jaja, ich weiß schon, dass die allermeisten von Ihnen nicht vorhaben, jemals auf einem Rechenschieber zu rechnen.
	      Zum Logarithmen-Üben ist das Ding (bzw. eine Vorstellung davon) aber immer noch praktisch!
	      Manchmal wollen auch Informatiker die Länge der Diagonale eines Quadrats berechnen — markiere dazu auf dem
	      Informatiker-Rechenschieber den Wert \(
	      \sqrt{2}
	      \).

	      Markiere nun noch denjenigen Wert, mit dem man die Kantenlänge eines Würfels multiplizieren muss,
	      um die Kantenlänge eines Würfels mit doppeltem Volumen zu erhalten.

	      Wie kann man mit unserem Rechenschieber — ohne die Skalen zu verlängern —
	      den Wert von \(
	      8 \cdot 512
	      \) ablesen?
	      (Tipp: am echten Rechenschieber heißt diese Technik Durchschieben. Bei unserm Modell ist ggf.
	      mal wieder die Näherung \(
	      2^{10} \approx 1000
	      \) hilfreich.)

	      Lösung:
	      \begin{enumerate}
		      \item
	      \end{enumerate}
	\item In wieviel Jahren hat sich eine mit einen Zinssatz von \(
	      p\%
	      \)im Jahr verzinste Geldanlage (incl. Zinseszins) verdoppelt?
	      Berechne diese Dauer für \(
	      p = 1,2,3,4
	      \) und \(
	      10
	      \) und vergleiche die Ergebnisse mit der Faustregel ”\(
	      70
	      \) Jahre geteilt durch Zinssatz“.

	      Lösung:
	      \begin{enumerate}
		      \item
	      \end{enumerate}
	\item Zeige, dass \(
	      ld 10
	      \) keine rationale Zahl ist (es gibt keine ganzen Zahlen \(
	      x , y
	      \) mit \(
	      ld 10 = x / y
	      \)).

	      Lösung:
	      \begin{enumerate}
		      \item
	      \end{enumerate}
\end{enumerate}
\end{document}