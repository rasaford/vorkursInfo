\documentclass[../main.tex]{subfiles}

\begin{document}
\begin{enumerate}
	\item Die Summe der ersten
	      \begin{math}
		      n
	      \end{math}
	      Zweierpotenzen beträgt:

	      \begin{math}
		      \sum_{i=0}^{n} 2^i = 2^{n+1} - 1
	      \end{math}

	      Da ein Schachbrett 64 Felder hat beträgt die Anzahl aller Reiskörner:

	      \begin{math}
		      \sum_{i=0}^{63} 2^i = 2^{64} - 1
	      \end{math}
	\item Das Seitenverähltnis beträgt:

	      \begin{math}
		      x = \frac{a}{b} = \frac{b}{
			      \frac{a}{2}
		      }
		      = \frac{2b}{a}
	      \end{math}

	      \begin{math}
		      \frac{a}{b} = \frac{2b}{a}
	      \end{math}
	      $|$
	      \begin{math}
		      \cdot a
	      \end{math}
	      $|$
	      \begin{math}
		      \cdot b
	      \end{math}

	      \begin{math}
		      \frac{a^2}{b^2} = 2
	      \end{math}
	      $|$
	      \begin{math}
		      \sqrt{}
	      \end{math}

	      \begin{math}
		      \frac{a}{b} = \sqrt{2}
	      \end{math}
	      \begin{math}
		      x = \frac{a}{b}
		      = \sqrt{2}
		      \approx 1,41
	      \end{math}

	      Somit können die Seitenlängen des Din A0 Blattes folgenermaßen
	      errechnet werden:

	      \begin{math}
		      1 m^2 = b \cdot \sqrt{2} \cdot b
	      \end{math}

	      \begin{math}
		      b = \sqrt{
			      \frac{1}{
				      \sqrt{2}
			      }
		      } m^2
		      \approx 0.841 m^2
	      \end{math}

	      \begin{math}
		      a = \sqrt{2} \cdot b
		      \approx 1,1891
	      \end{math}
	      \begin{itemize}
		      \item DIN A0
		            lets have some fun here
		            \begin{math}
			            a = 0.841
		            \end{math}

		            \begin{math}
			            b = \frac{
				            \sqrt{2} \cdot 0.841
			            }{2}
			            = 0.595
		            \end{math}
	      \end{itemize}
\end{enumerate}
\end{document}