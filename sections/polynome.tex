\documentclass[../main.tex]{subfiles}
\usepackage{amsmath}
\usepackage{listings}

\begin{document}
\begin{enumerate}
	\item Berechne \begin{math}
		      2^n
	      \end{math} für \begin{math}
		      0 \dots 20
	      \end{math}.

	      Für größere Zweierpotenzen ist die Faustregel "\begin{math}
		      2^{10}
	      \end{math} oder \begin{math}
		      1000
	      \end{math} - das ist doch praktisch das selbe" nützlich.
	      Gib damit Näherungen für \begin{math}
		      2^{32}
	      \end{math} und \begin{math}
		      2^{64}
	      \end{math} an.

	      Lösung:
	      \begin{enumerate}
		      \item
	      \end{enumerate}
	\item Gegeben sind die Funktionen \begin{math}
		      f(x) = 6 \cdot x^2
	      \end{math} und \begin{math}
		      g(x) = 2 \cdot x^3
	      \end{math}.
	      \begin{enumerate}
		      \item Skizziere beide Graphen.
		      \item Für welche \begin{math}
			            x
		            \end{math} ist \begin{math}
			            f(x) = g(x)
		            \end{math}?
		            Für welche ist \begin{math}
			            f(x) > g(x)
		            \end{math} und für welche \begin{math}
			            f(x) < g(x)
		            \end{math}?
	      \end{enumerate}
	\item Für welche ganzen Zahlen \begin{math}
		      n
	      \end{math} ist \begin{math}
		      2^n > n^2
	      \end{math}? (Probieren ist hier besser als rechnen!)

	      Lösung:
	      \begin{enumerate}
		      \item
	      \end{enumerate}
	\item \begin{enumerate}
		      \item Skizziere den Graph der Funktion \begin{math}
			            x \mapsto 2^x
		            \end{math} für \begin{math}
			            x = -1000 \dots 10
		            \end{math} und diskutiere den Satz ”die Exponentialfunktion ist ein rechter Winkel“.
		      \item Bestimme die kleinste Zahl \begin{math}
			            x_0
		            \end{math}, so dass für alle \begin{math}
			            x \geq x_0
		            \end{math} gilt: \begin{math}
			            2^x \geq 16x^3
		            \end{math}.
		      \item Wie ändert sich die Antwort in b),
		            wenn die rechte Seite \begin{math}
			            (16x^3)
		            \end{math} mit \begin{math}
			            2^{13} = 8192
		            \end{math} multipliziert wird, also die Ungleichung \begin{math}
			            2^x \geq 131072 x^3
		            \end{math} betrachtet wird?
	      \end{enumerate}

	      Lösung:
	      \begin{enumerate}
		      \item
	      \end{enumerate}
	\item Wie viele verschiedene Zustände kann man mit \begin{math}
		      n
	      \end{math} Bits darstellen?
	      Speziell: wenn wir ganze Zahlen (bei 0 beginnend) in 32 Bit speichern,
	      wie weit können wir damit zählen?

	      Lösung:
	      \begin{enumerate}
		      \item
	      \end{enumerate}
	\item
	      Vereinfache folgende Therme (dabei seien \begin{math}
		      x, y, z > 0
	      \end{math}):
	      \begin{enumerate}
		      \item \begin{math}
			            \sqrt[5]{2^{15}}
		            \end{math}
		      \item \begin{math}
			            (\frac{8}{125})^{-\frac{1}{3}}
		            \end{math}
		      \item \begin{math}
			            \sqrt{\sqrt[3]{x}}
		            \end{math}
		      \item \begin{math}
			            (\sqrt[3]{x} \cdot \sqrt{y^3})^6
		            \end{math}
		      \item \begin{math}
			            \frac{(x^2 \cdot y^3 z^4)^2}{
				            (x \cdot y \cdot z)^{-2}
			            }
		            \end{math}
		      \item \begin{math}
			            \frac{x - y}{ \sqrt{x} - \sqrt{y} }
		            \end{math}
	      \end{enumerate}

	      Lösung:
	      \begin{enumerate}
		      \item \begin{math}
			            \sqrt[5]{2^{15}}
			            =
		            \end{math}
		      \item \begin{math}
			            (\frac{8}{125})^{-\frac{1}{3}}
			            =
		            \end{math}
		      \item \begin{math}
			            \sqrt{\sqrt[3]{x}}
			            =
		            \end{math}
		      \item \begin{math}
			            (\sqrt[3]{x} \cdot \sqrt{y^3})^6
			            =
		            \end{math}
		      \item \begin{math}
			            \frac{(x^2 \cdot y^3 z^4)^2}{
				            (x \cdot y \cdot z)^{-2}
			            }
			            =
		            \end{math}
		      \item \begin{math}
			            \frac{x - y}{ \sqrt{x} - \sqrt{y} }
			            =
		            \end{math}
	      \end{enumerate}
	\item Um eine Koch-Kurve zu konstruieren, beginnen wir mit einer Strecke der Länge 1 und
	      ersetzen nun in jeder Runde jede bis dahin erzeugte Strecke durch vier Teilstrecken von je einem Drittel
	      der Länge gemäß folgendem Muster

	      Die Ergebnisse der Runden zwei bis fünf sehen dann so aus
	      (die Koch-Kurve selbst ist das fraktale Objekt, das im Grenzprozess unendlich vieler Iterationen entsteht):

	      Schätze die Länge dieser Streckenzüge! Wie lang sind sie wirklich?

	      Lösung:
	      \begin{enumerate}
		      \item
	      \end{enumerate}
	\item Lineare Gleichungen - bestimme für die folgenden Gleichungen jeweils alle
	      \begin{math}
		      x
	      \end{math}, die die Gleichung erfüllen:
	      \begin{enumerate}
		      \item \begin{math}
			            4 \cdot ( x - 1)
			            = 5 \cdot (x - 2)
		            \end{math}
		      \item \begin{math}
			            \frac{1}{x-1}
			            = \frac{x+1}{x-2} -1
		            \end{math}
		      \item \begin{math}
			            (x + 2) \cdot (x- 2) = 21
		            \end{math}

		            Naja, die letzte Gleichung ist nicht linear in \begin{math}
			            x
		            \end{math}; wen das stört, der führt
		            zwischendrin ein \begin{math}
			            y := x^2
		            \end{math} ein\ldots
	      \end{enumerate}

	      Lösung:
	      \begin{enumerate}
		      \item
	      \end{enumerate}
	\item Leite die Lösungsformel \begin{math}
		      x_{1,2} = - \frac{p}{2} \pm \sqrt{ \frac{p^2}{4} - q }
	      \end{math} der quadratischen Gleichung mit Hilfe der so genannten quadratischen Ergänzung her,
	      d.h. bringe die Gleichung \begin{math}
		      x^2 + px + q = 0
	      \end{math} erst in die Form \begin{math}
		      (x + \alpha)^2 + \beta = 0
	      \end{math} und löse die Gleichung dann nach \begin{math}
		      x
	      \end{math} auf.

	      Lösung:
	      \begin{enumerate}
		      \item
	      \end{enumerate}
	\item Gegeben sind die Punkte \begin{math}
		      A(0|2), B(2|6)
	      \end{math} und \begin{math}
		      C(-1|1.5)
	      \end{math}.
	      \begin{enumerate}
		      \item Konstruiere eine Funktion \begin{math}
			            f(x) = ax^2 + bx + c
		            \end{math}, so dass ihr Graph durch
		            diese drei Punkte verläuft.
		            Wie viele solcher Funktionen gibt es?
		      \item Bestimme \begin{math}
			            y_1
		            \end{math} und \begin{math}
			            y_2
		            \end{math} so, dass die Punkte
		            \begin{math}
			            D(4|y_1)
		            \end{math} und \begin{math}
			            E(-3|y_2)
		            \end{math} ebenfalls auf dem Graphen liegen!
	      \end{enumerate}

	      Lösung:
	      \begin{enumerate}
		      \item
	      \end{enumerate}
	\item Dividiere \begin{math}
		      x^5 - x^4 + 2x^3 -2x2 -8x +8
	      \end{math} durch \begin{math}
		      x^2 - 2
	      \end{math} und bestimme alle Nullstellen von \begin{math}
		      x^5 - x^4 + 2x^3 - 2x^2 - 8x +8
	      \end{math}.

	      Lösung:
	      \begin{enumerate}
		      \item
	      \end{enumerate}
	\item Berechne \begin{math}
		      (\sum_{i = 0}^{n} x^i) \cdot ( x -1)
	      \end{math} und stelle damit eine geschlossene Formel (d.h. ohne Summenzeichen) zur Berechnung von \begin{math}
		      \sum_{ i = 0}^{n} x^i
	      \end{math} für \begin{math}
		      x \neq 1
		\end{math} auf.

	      Lösung:
	      \begin{enumerate}
		      \item
	      \end{enumerate}
\end{enumerate}
\end{document}