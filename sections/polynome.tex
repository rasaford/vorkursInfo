\documentclass[../main.tex]{subfiles}
\usepackage{amsmath} 
\usepackage{multicol}
\usepackage{listings}

\begin{document}
\begin{enumerate}
	\item Berechne \( 2^n \) für \( 0 \dots 20 \).

	      Für größere Zweierpotenzen ist die Faustregel "`\( 2^{10} \) oder \( 1000 \) - das ist doch praktisch das selbe"'
	      nützlich Gib damit Näherungen für \( 2^{32} \) und \( 2^{64} \) an.

	      Lösung:
	      \begin{multicols}{3}
		      \begin{enumerate}
			      \item \( 2^0 = 1 \)
			      \item \( 2^1 = 2 \)
			      \item \( 2^2 = 4 \)
			      \item \( 2^3 = 8 \)
			      \item \( 2^4 = 16 \)
			      \item \( 2^5 = 32 \)
			      \item \( 2^6 = 64 \)
			      \item \( 2^7 = 128 \)
			      \item \( 2^8 =  256 \)
			      \item \( 2^9 = 512 \)
			      \item \( 2^{10} = 1024 \)
			      \item \( 2^{11} = 2048 \)
			      \item \( 2^{12} = 4096 \)
			      \item \( 2^{12} = 8192 \)
			      \item \( 2^{13} = 16384 \)
			      \item \( 2^{14} = 32767 \)
			      \item \( 2^{15} = 65535 \)
			      \item \( 2^{16} = 131071 \)
			      \item \( 2^{17} = 262143 \)
			      \item \( 2^{18} = 524287 \)
			      \item \( 2^{19} = 1 048 575 \)
			      \item \( 2^{20} = 2 097 151 \)
		      \end{enumerate}
	      \end{multicols}
	\item Gegeben sind die Funktionen \( f(x) = 6 \cdot x^2 \) und \( g(x) = 2 \cdot x^3 \).
	      \begin{enumerate}
		      \item Skizziere beide Graphen.
		      \item Für welche \( x \) ist \( f(x) = g(x) \)?
		            Für welche ist \( f(x) > g(x) \) und für welche \( f(x) < g(x) \)?
	      \end{enumerate}

	      Lösung:
	      \begin{enumerate}
		      \item
		      \item Schnittpunkte der beiden Funktionen durch Berechnen
		            der Nullstellen von \( f(x) = g(x) \)

		            \( 6x^2 = 2^3 \)

		            \( x_{1,2} = 0 \lor x_3 = 3 \)

		            Durch Einsetzen von Werten um die Nullstellen der Funktion kann man die größere der
		            beiden bestimmen.

		            \( f(2.5) = 37.5 \)

		            \( g(2.5) = 31.25\)

		            Somit gilt:

		            Für \( -\infty \leq x \leq 3 \) ist \( f(x) \geq g(x) \)

		            Für \( 3 < x \leq \infty \) ist \( g(x) \geq f(x) \)
	      \end{enumerate}
	\item Für welche ganzen Zahlen \( n \) ist \( 2^n > n^2 \)?
	      (Probieren ist hier besser als rechnen!)

	      Lösung:
	      \begin{enumerate}
		      \item Durch einsetzen von Werten in die Funktionen können die Schnittpunkte bestimmt werden

		            \[ \begin{array}{c|cccccccc}
				                & -1    & 0 & 1 & 2 & 3 & 4  & 5  & 6  \\
				            \hline
				            2^n & 1 / 2 & 1 & 2 & 4 & 8 & 16 & 32 & 64 \\
				            n^2 & 1     & 0 & 1 & 4 & 9 & 16 & 25 & 36 \\
			            \end{array} \]

		            Für \( -\infty \leq x \leq 0 : n^2 > 2^n \)

		            Für \( 0 < x \leq 2 : 2^n > n^2 \)

		            Für \( 2 < x \leq 4 : n^2 > 2^n \)

		            Für \( 4 < x \leq \infty : 2^n > n^2 \)

	      \end{enumerate}
	\item \begin{enumerate}
		      \item Skizziere den Graph der Funktion \( x \mapsto 2^x \) für \(Rx = -1000 \dots 10 \)
		            und diskutiere den Satz ”die Exponentialfunktion ist ein rechter Winkel“.
		      \item Bestimme die kleinste Zahl \( x_0 \), so dass für alle \( x \geq x_0 \)
		            gilt: \( 2^x \geq 16x^3 \).
		      \item Wie ändert sich die Antwort in b),
		            wenn die rechte Seite \( (16x^3) \) mit \( 2^{13} = 8192 \)
		            multipliziert wird, also die Ungleichung \( 2^x \geq 131072 x^3 \) betrachtet wird?
	      \end{enumerate}

	      Lösung:
	      \begin{enumerate}
		      \item
		      \item \( 2^n \geq 16x^3 \)

		            \( 2^n \geq 2^4x^3 \) \( | \div 2^4 \)

		            \( 2^{n-4} \geq x^3 \)

		            \[ \begin{array}{c|ccc}
				            x         & 15   & 16   & 17   \\
				            \hline
				            2^{x - 4} & 2048 & 4096 & 8192 \\
				            x^3       & 3375 & 4096 & 4913 \\
			            \end{array} \]

		            Die kleinste Zahl, sodass \( 2^x \geq 16x^3 \) ist \( 16 \).
		      \item
	      \end{enumerate}
	\item Wie viele verschiedene Zustände kann man mit \( n \) Bits darstellen?
	      Speziell: wenn wir ganze Zahlen (bei 0 beginnend) in 32 Bit speichern,
	      wie weit können wir damit zählen?

	      Lösung:
	      \begin{enumerate}
		      \item In \( n  \) bits können \( 2^n \) Werte dargestellt werden.
		            Somit kann bis \( 2^n - 1 \) gezählt werden.
	      \end{enumerate}
	\item
	      Vereinfache folgende Therme (dabei seien \( x, y, z > 0 \)):
	      \begin{enumerate}
		      \item \( \sqrt[5]{2^{15}} \)
		      \item \( (\frac{8}{125})^{-\frac{1}{3}} \)
		      \item \( \sqrt{\sqrt[3]{x}} \)
		      \item \( (\sqrt[3]{x} \cdot \sqrt{y^3})^6 \)
		      \item \( \frac{(x^2 \cdot y^3 z^4)^2}{
			            (x \cdot y \cdot z)^{-2}
		            } \)
		      \item \( \frac{x - y}{ \sqrt{x} - \sqrt{y} } \)
	      \end{enumerate}

	      Lösung:
	      \begin{enumerate}
		      \item \( \sqrt[5]{2^{15}}
			            = 2^{15/5} = 2^3 = 8
		            \)
		      \item \( (\frac{8}{125})^{-\frac{1}{3}}
		            = \frac{1}{ (\frac{8}{125})^{ \frac{1}{3} }}
		            = \frac{1}{ \sqrt[3]{ \frac{8}{125} }}
		            = \frac{1}{ \frac{2}{5}}
		            = \frac{5}{2}
		            = 2.5
		            \)
		      \item \( \sqrt{\sqrt[3]{x}}
		            = \Big( x^{ \frac{1}{3} } \Big)^{ \frac{1}{2} }
		            = x^{ \frac{1}{6}}
		            = \sqrt[6]{x}
		            \)
		      \item \( (\sqrt[3]{x} \cdot \sqrt{y^3})^6
		            = x^{ \frac{1}{3} \cdot 6} \cdot y^{ \frac{3}{2} \cdot 6 }
		            = x^2 \cdot y^9
		            \)
		      \item \( \frac{(x^2 \cdot y^3 z^4)^2}{
			            (x \cdot y \cdot z)^{-2}
		            }
		            = (x^2 \cdot y^3 \cdot z^4)^2 \cdot (xyz)^2
		            = (x^3y^4z^5)^2
		            = x^6y^8z^{10}
		            \)
		      \item \( \frac{x - y}{ \sqrt{x} - \sqrt{y} }
		            = \frac{ (\sqrt{x} + \sqrt{y})(\sqrt{x} - \sqrt{y}) }{
			            \sqrt{x} - \sqrt{y}}
		            = \sqrt{x} + \sqrt{y} \)
	      \end{enumerate}
	\item Um eine Koch-Kurve zu konstruieren, beginnen wir mit einer Strecke der Länge 1 und
	      ersetzen nun in jeder Runde jede bis dahin erzeugte Strecke durch vier Teilstrecken von je einem Drittel
	      der Länge gemäß folgendem Muster

	      Die Ergebnisse der Runden zwei bis fünf sehen dann so aus
	      (die Koch-Kurve selbst ist das fraktale Objekt, das im Grenzprozess unendlich vieler Iterationen entsteht):

	      Schätze die Länge dieser Streckenzüge! Wie lang sind sie wirklich?

	      Lösung:
	      \begin{enumerate}
		      \item Die Länge bei den aufeinander folgenden Iterationen beträgt:

		            \(  l_0 = 1 \)

		            \( l_1 = 4 \cdot \frac{1}{3} l_0 \)

		            \( l_2 = 4 \cdot \frac{1}{3} (4 \cdot \frac{1}{3} l_0 )
		            =  \frac{4}{3} \cdot \frac{4}{3} l_0
		            = ( \frac{4}{3} )^2 l_0 \)

		            Somit kann die Länge der Kurve bei Iteration \( n \) durch
		            \( ( \frac{4}{3})^n \) berechnet werden.

		            Somit ist die Länge der Kurve nicht begrenzt.
		            \[ \lim_{n \to \infty} \Big( \frac{4}{3} \Big)^n = \infty  \]
	      \end{enumerate}
	\item Lineare Gleichungen - bestimme für die folgenden Gleichungen jeweils alle
	      \( x \), die die Gleichung erfüllen:
	      \begin{enumerate}
		      \item \( 4 \cdot ( x - 1)
		            = 5 \cdot (x - 2)
		            \)
		      \item \( \frac{1}{x-1}
		            = \frac{x+1}{x-2} -1
		            \)
		      \item \( (x + 2) \cdot (x- 2) = 21
		            \)

		            Naja, die letzte Gleichung ist nicht linear in \( x \);
		            wen das stört, der führt
		            zwischendrin ein \( y := x^2 \) ein\ldots
	      \end{enumerate}

	      Lösung:
	      \begin{enumerate}
		      \item \( 4 \cdot ( x - 1) = 5 \cdot (x - 2) \)

		            \( 4x-4 = 5x - 10 \) \( | +4 | -4x \)

		            \( 0 = x - 6 \)  \( | +6  \)

		            \( 6 = x \)
		      \item \( \frac{1}{x-1} = \frac{x+1}{x-2} -1 \)  \( | \cdot (x-1) \)

		            \(  1 = \frac{ (x+1)(x-1) }{ x-2 } - (x - 1) \) \( | \cdot (x - 2) \)

		            \( x - 2 = (x + 1)(x - 1) - (x - 1) (x - 2) \)

		            \( x-2 = x^2 - 1 - x^2 -2x -x +2 \)

		            \( x  = -3x -1  \) \( | -x  \)

		            \( 0 = -4x -1  \) \( | +1 \)

		            \( 1 = -4x \) \( | \div (-4) \)

		            \( x = -\frac{1}{4} \)
		      \item \( (x + 2) \cdot (x- 2) = 21 \)

		            \( x^2 - 16 = 21 \) \( | +16 \)

		            \( x^2 = 5  \) \( |\sqrt{} \)

		            \( x_{1,2} = \pm \sqrt{5} \)
	      \end{enumerate}
	\item Leite die Lösungsformel \( x_{1,2} = - \frac{p}{2} \pm \sqrt{ \frac{p^2}{4} - q } \)
	      der quadratischen Gleichung mit Hilfe der so genannten quadratischen Ergänzung her,
	      d.h. bringe die Gleichung \( x^2 + px + q = 0 \) erst in die Form
	      \( (x + \alpha)^2 + \beta = 0 \) und löse die Gleichung dann nach \( x \) auf.

	      Lösung:
	      \begin{enumerate}
		      \item \( x^2 + px + q = 0 \)

		            \( x^2 + px + \frac{p^2}{4} - \frac{p^2}{4} + q = 0\)

		            \( (x + \frac{p}{2} )^2 - \frac{p^2}{4} + q = 0 \)  \( |+ \frac{p^2}{4} | - q \)

		            \( (x + \frac{p}{2} )^2 = \frac{p^2}{4} - q \) \( |\sqrt{} \)

		            \( x + \frac{p}{2} = \pm \sqrt{ \frac{p^2}{4} - q }\) \( |- \frac{p}{2} \)

		            \( x_{1,2} = -\frac{p}{2} \pm \sqrt{ \frac{p^2}{4} - q }\) \( |- \frac{p}{2} \)
	      \end{enumerate}
	\item Gegeben sind die Punkte \( A(0|2), B(2|6) \) und \( C(-1|1.5) \).
	      \begin{enumerate}
		      \item Konstruiere eine Funktion \( f(x) = ax^2 + bx + c \), so dass ihr Graph durch
		            diese drei Punkte verläuft.
		            Wie viele solcher Funktionen gibt es?
		      \item Bestimme \( y_1 \) und \( y_2 \) so, dass die Punkte
		            \( D(4|y_1) \) und \( E(-3|y_2) \) ebenfalls auf dem Graphen liegen!
	      \end{enumerate}

	      Lösung:
	      \begin{enumerate}
		      \item Durch die \( 3 \) gegebenen Punkte können die \( 3 \) folgenden Funktionen
		            definiert werden, durch welche die ursprüngliche Funktion \( f(x) \) eindeutig bestimmt werden kann.
		            \[ a \cdot 0^2 + b \cdot 0 + c = 2 \]
		            \[ a \cdot 2^2 + b \cdot 2 + c = 6 \]
		            \[ a \cdot (-1)^2 + b \cdot (-1) + c = 1.5 \]

		            Diese Gleichungen können in ein LGS umgeschrieben und dann mit dem
		            Gaus'schen Algorithmus gelöst werden.

		            \( \begin{array}{ccc|c}
			            0 & 0  & 1 & 2   \\
			            4 & 2  & 1 & 6   \\
			            1 & -1 & 1 & 1.5 \\
		            \end{array}
		            \rightarrow
		            \begin{array}{ccc|c}
			            1 & 0 & 0 & 0.5 \\
			            0 & 1 & 0 & 1   \\
			            0 & 0 & 1 & 2   \\
		            \end{array} \)

		            Es können nun die gelösten Parameter in die ursprüngliche Funktionsgleichung
		            \( f(x) = ax^2 + bx + c \) eingesetzt werden.
		            \[ f(x) = 0.5 x^2 + x + 2 \]

		      \item \( f(4) = 0.5 \cdot 4^2 + 4 + 2 = 14  \Rightarrow D(4|14) \)

		            \( f(-3) = 0.5 \cdot (-3)^2 -3 + 2 = 3.5  \Rightarrow E(-3|3.5) \)
	      \end{enumerate}
	\item Dividiere \( x^5 - x^4 + 2x^3 -2x2 -8x +8 \) durch \( x^2 - 2 \)
	      und bestimme alle Nullstellen von \( x^5 - x^4 + 2x^3 - 2x^2 - 8x +8 \).

	      Lösung:
	      \begin{enumerate}
		      \item \( x^5 - x^4 + 2x^3 -2x2 -8x +8  \div (x^2 -2) = x^3 -x^2 + 4x +4 \)

		            Es muss eine der Nullstellen der Funktion erraten werden \( (x_1=1) \).
		            Diese kann dann abgespalten werden.

		            \(  x^5 - x^4 + 2x^3 -2x2 -8x +8 \div (x - 1) =  x^4 + 2x^2 - 8 \)

		            Substitution mit \( x^2 = y \)

		            \( y^2 + 2y - 8 = 0 \)

		            \( y_{1,2} = -1 \pm \sqrt{ \frac{4}{4} + 8} \)

		            \( y{1,2} = -1 \pm 3 \)

		            \(  y_1 = -4 \lor y_2 = 2 \)

		            Resubstitution mit \( y = x^2 \)
		            \begin{multicols}{2}
			            \begin{itemize}
				            \item[] \( -4 = x^2  | \sqrt{} \)

				                  \( x_{2,3} = \pm \sqrt{-4} \)
				            \item[] \( 2 = x^2 | \sqrt{} \)

				                  \( x_{4,5} = \pm \sqrt{2} \)
			            \end{itemize}
		            \end{multicols}
	      \end{enumerate}
	\item Berechne \( (\sum_{i = 0}^{n} x^i) \cdot ( x -1) \)
	      und stelle damit eine geschlossene Formel (d.h. ohne Summenzeichen) zur Berechnung von
	      \(\sum_{ i = 0}^{n} x^i \) für \( x \neq 1 \) auf.

	      Lösung:
	      \begin{enumerate}
		      \item \( \sum_{i = 0}^{n} x^i \cdot (x - 1) \)

		            \( = x \cdot \sum_{i = 0}^{n} x^i - \sum_{i = 0}^{n} x^i \)

		            \( = x + x^2 + x^3 + \dots + x^{n+1} - (1 + x + x^2 + \dots + x^n) \)

		            \( = -1 + x^{n+1} \)

			\item \( \sum_{i = 0}^{n} x^i \cdot (x - 1) = x^{n+1} -1  | \div (x-1) \)

		            \( \sum_{i = 0}^{n} x^i = \frac{ x^{n + 1} - 1 }{ x - 1 }  \)
	      \end{enumerate}
\end{enumerate}
\end{document}