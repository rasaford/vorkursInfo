\documentclass[../main.tex]{subfiles}
	\usepackage{amsmath}
	\usepackage{amsfonts}
	\usepackage{listings}

\begin{document}
\begin{enumerate}
	\item Der Legende nach gewährte einst Sher Khan, der König von Indien, dem Erfinder des Schachspiels für
	      die Erfindung dieses außergewöhnlichen Spiels die Gunst, sich seine Belohnung selbst aussuchen zu durfen.
	      Der bescheidene (?) Erfinder verlangte lediglich ein paar Reiskörner.
	      Und zwar sollten auf jedem Feld des Schachbrett jeweils doppelt so viele Körner,
	      wie auf dem vorhergehenden Feld liegen (für Mathematiker:
	      \begin{math}
		      1 + 2 + 2^2 + 2^3 + \dots
	      \end{math}
	      Reiskörner). Auf wie viele Reiskörner hätte sich die Belohnung belaufen?

	      Überlege Dir dazu allgemein, welche Zahlen herauskommen, wenn man die
	      Zweierpotenzen aufsummiert. Studierende mit Nebenfach Physik schätzen
	      bitte die Zahl der Reiskörner in Tonnen (Studierende mit Nebenfach Agrar-
	      wissenschaften entsprechend in Doppelzentner).

	      Lösung:
	      \begin{enumerate}
		      \item
		            \begin{math}
			            \sum_{i=0}^{4} 2^i = 1 + 2 + 4 + 8 + 16 = 31 = 2^5 - 1
		            \end{math}

		            Die Summe der ersten
		            \begin{math} n \end{math}
		            Zweierpotenzen beträgt:

		            \begin{math}
			            \sum_{i=0}^{n} 2^i = 2^{n+1} - 1
		            \end{math}

		            Da ein Schachbrett \begin{math}
			            64
		            \end{math} Felder hat beträgt die Anzahl aller Reiskörner:

		            \begin{math}
			            \sum_{i=0}^{63} 2^i = 2^{64} - 1 = 18446744073709551615
		            \end{math}
	      \end{enumerate}
	\item Die Papiergrößen nach DIN sind so gebaut, dass man durch Falten in der
	      Mitte der langen Seite die nächstkleinere Größe bekommt. Also: hat man ein
	      Papier der  Höhe \begin{math} a \end{math}
	      und der Breite
	      \begin{math}
		      b
	      \end{math}
	      (in Hochformat, also \begin{math}
		      a > b
	      \end{math}),
	      dann ist
	      das kleinere Papier \begin{math}
		      a
	      \end{math} breit und \begin{math}
		      b
	      \end{math} hoch. Zusätzlich gilt aber bei den DIN-Größen,
	      dass das Seitenverhältnis dabei gleich bleibt:
	      \begin{math}
		      a : b = b : \frac{a}{b}
	      \end{math} Wie groß ist demnach das Seitenverhältnis
	      \begin{math}
		      x = \frac{a}{b}
	      \end{math}?

	      Ein Blatt DIN A0 hat die Fläche \begin{math}
		      1 m^2
	      \end{math}
	      Wie hoch und wie breit ist es?
	      Wie hoch und wie breit ist ein Blatt DIN A4?

	      Lösung:
	      \begin{enumerate}
		      \item Das Seitenverhältnis beträgt:

		            \begin{math}
			            x = \frac{a}{b}
			            = \frac{b}{
				            \frac{a}{2}
			            }
			            = \frac{2b}{a}
		            \end{math}

		            Somit gilt:

		            \begin{math}
			            \frac{a}{b} = \frac{2b}{a}
		            \end{math}
		            \(|\)
		            \begin{math}
			            \cdot a
		            \end{math}
		            \(|\)
		            \begin{math}
			            \cdot b
		            \end{math}

		            \begin{math}
			            \frac{a^2}{b^2} = 2
		            \end{math}
		            \(|\)
		            \begin{math}
			            \sqrt{}
		            \end{math}

		            \begin{math}
			            \frac{a}{b}
			            = \sqrt{2}
			            = x
			            \approx 1,41
		            \end{math}

		            Somit können die Seitenlängen des Din A0 Blattes folgenermaßen
		            errechnet werden:

		            \begin{math}
			            1 m^2
			            = (b) \cdot (\sqrt{2} \cdot b)
		            \end{math}

		            \begin{math}
			            b = \sqrt{
				            \frac{1}{\sqrt{2}} m^2
			            }
			            \approx 0.841 m
		            \end{math}

		            \begin{math}
			            a = \sqrt{2} \cdot b
			            \approx 1,1891 m
		            \end{math}

		            Es gilt:

		            \begin{math}
			            a_{neu} = b_{alt}
		            \end{math}

		            \begin{math}
			            b_{neu} = \frac{ \sqrt{2} \cdot a_{neu}}{ 2 }
		            \end{math}

		            Somit können jetzt alle Maße der verschiedenen DIN Größen berechnet werden:

		            \begin{math}
			            \begin{array}{cccccc}
				              & A0       & A1      & A2       & A3       & A4       \\
				            \hline
				            a & 1.1891 m & 0.841 m & 0.595 m  & 0.4205 m & 0.2975 m \\
				            b & 0.841 m  & 0.595 m & 0.4205 m & 0.2975 m & 0.2103 m \\
			            \end{array}
		            \end{math}
	      \end{enumerate}
	\item Viel Schöner würden die Papiergrößen nach dem so genannten "goldenen Schnitt“ ausschauen.
	      Dazu muss sich \begin{math}
		      a : b
	      \end{math} verhalten wie \begin{math}
		      b : (a - b)
	      \end{math}. Berechne wieder das Seitenverhältnis \begin{math}
		      x = \frac{a}{b}
	      \end{math}. Wer eine Schere dabei hat,
	      kann anschließend Wahrheitswert des ersten Satzes überprüfen.

	      Lösung:
	      \begin{enumerate}
		      \item Das Seitenverhältnis lässt sich auch schreiben als:

		            \begin{math}
			            x
			            = \frac{a}{b}
			            = \frac{b}{a-b}
		            \end{math}

		            \begin{math}
			            \frac{b}{a}
			            = \frac{a-b}{b}
		            \end{math}

		            \begin{math}
			            \frac{1}{x}
			            = \frac{a}{b} - \frac{b}{b}
		            \end{math}

		            \begin{math}
			            \frac{1}{x}
			            = x - 1
		            \end{math}
		            \(|\) \begin{math}
			            \cdot x
		            \end{math}

		            \begin{math}
			            1
			            = x^2 - x
		            \end{math}
		            \(|\) \begin{math}
			            -1
		            \end{math}

		            \begin{math}
			            0 = x^2 - x - 1
		            \end{math}

		            Von diesem Polynom können dann die Nullstellen bestimmt werden, um den goldenen Schnitt zu erhalten:

		            \begin{math}
			            x_{1,2} = \frac{1}{2} \pm \sqrt{
				            \frac{1}{4} + 1
			            }
		            \end{math}

		            \begin{math}
			            x_1 = \frac{1 - \sqrt{5}}{2} \lor
			            x_2 = \frac{1 + \sqrt{5}}{2} = \varphi
		            \end{math}
	      \end{enumerate}
	\item Interessant sind die Koeffizienten, die herauskommen, wenn man die Terme \begin{math}
		      (x + 1)^n
	      \end{math} ausmultipliziert (also z.B. \begin{math}
		      (x + 1)^2 = x^2 + 2x + 1
	      \end{math}).
	      Berechne dies für die ersten paar \begin{math}
		      n
	      \end{math} und überlege, nach welchem Gesetz die Koeffizienten gebildet werden.

	      Lösung:
	      \begin{enumerate}
		      \item Beim ausmultiplizieren der einzelnen Terme sind die Dreieckszahlen in den Koeffizienten erkennbar.

		            \begin{math}
			            (x + 1)^2 = 1 \cdot x^2 + 2 \cdot x + 1 \cdot 1
		            \end{math}

		            \begin{math}
			            (x + 1)^3 = 1 \cdot x^3 + 3 \cdot x^2 + 3 \cdot x + 1 \cdot 1
		            \end{math}

		            Somit kann ein solcher Therm mit einer beliebigen Potenz folgendermaßen dargestellt werden:

		            \begin{math}
			            (x + 1)^n = \sum_{i = 0}^{n} \binom{n}{i} \cdot x^{n-i} \cdot 1^i
		            \end{math}

		            Diese Formel lässt sich auf einen beliegen Therm mit den Variablen \begin{math}
			            a, b
		            \end{math} verallgemeinern.

		            \begin{math}
			            (a + b)^n = \sum_{i = 0}^{n} \binom{n}{i} \cdot a^{n-i} \cdot b^i
		            \end{math}
	      \end{enumerate}
	\item Ein Vater ist heute \begin{math}
		      a
	      \end{math} Jahre älter als sein Sohn.
	      In \begin{math}
		      b
	      \end{math} Jahren wird er \begin{math}
		      c
	      \end{math} Jahre älter als \begin{math}
		      d
	      \end{math}-mal so alt sein wie sein Sohn heute.
	      Wie alt sind Vater und Sohn gegenwärtig?

	      Führen alle Parameterwerte \begin{math}
		      a, b, c, d
	      \end{math} zu einer Lösung?
	      Welche davon sind sinnvoll?

	      Lösung:
	      \begin{enumerate}
		      \item
		            \begin{math}
			            v :=
		            \end{math} Alter des Vaters heute,
		            \begin{math}
			            s :=
		            \end{math} Alter des Sohns heute

		            Aus dem Text erhält man:

		            \begin{math}
			            v = s + a
		            \end{math}

		            \begin{math}
			            c = (v + b) - d \cdot s
		            \end{math}
		            Einsetzen und vereinfachen:

		            \begin{math}
			            c = s + a + b - d \cdot s
		            \end{math}
		            \(|\)\begin{math}
			            - a
		            \end{math}
		            \(|\)\begin{math}
			            - b
		            \end{math}

		            \begin{math}
			            c - a - b = s - d \cdot s
		            \end{math}

		            \begin{math}
			            c - a - b = s \cdot (1 - d)
		            \end{math}
		            \(|\)\begin{math}
			            \div (1-d)
		            \end{math}

		            \begin{math}
			            s = \frac{c - a - b}{1-d}
		            \end{math} für \begin{math}
			            d \neq 1
		            \end{math}

		            \begin{math}
			            v = s + a = \frac{c - a - b}{1-d} + a = \frac{c - b - a \cdot d}{1 - d}
		            \end{math}
	      \end{enumerate}
	\item Der Chinese Xu Yue stellte gegen 190 n. Chr. das folgende Problem:
	      Wie viele Hähne, Hennen und Küken kann man für 100 Münzen kaufen,
	      wenn man insgesamt 100 Tiere haben will und ein Hahn 5 Münzen,
	      eine Henne 4 Münzen und 4 Küken eine Münze kosten?
	      Die 100 Münzen sollen dabei vollständig verbraucht werden.

	      Lösung:
	      \begin{enumerate}
			  \item 
			 \begin{math}
				 a := 
			 \end{math} Hahn, 
			 \begin{math}
				 b := 
			 \end{math} Henne,
			 \begin{math}
				 c := 
			 \end{math} Küken

			  Da man insgesamt genau 100 Tiere haben möchte, muss gelten:
					
		            \begin{math}
			            1 \cdot a + 1 \cdot b + 1 \cdot c = 100
		            \end{math}

		            Dazu sollen genau 100 Münzen verbraucht werden, somit muss außerdem gelten:

		            \begin{math}
			            5 \cdot a + 4 \cdot b + \frac{c}{4} = 100
		            \end{math}

		            Diese zwei Gleichungen können in ein lineares Gleichungssystem geschrieben werden.
		            Da dies aber 3 Variablen mit nur 2 Gleichungen besitzt ist es nicht eindeutig lösbar.

		            \begin{math}
			            \begin{array}{ccc|c}
				            5 & 4 & 4 & 100 \\
				            1 & 1 & 1 & 100 \\
			            \end{array}
			            \rightarrow
			            \begin{array}{ccc|c}
				            19 & 15 & 0 & 300 \\
				            1  & 1  & 1 & 100 \\
			            \end{array}
		            \end{math}

		            \begin{math}
			            b = \frac{300 - 19a}{15}
			            = 20 - \frac{19a}{15}
		            \end{math}

		            \begin{math}
						c = - a - b + 100
						= - a - (20 - \frac{19a}{15}) + 100 = \frac{4a}{15} + 80
					\end{math}
					
					Ganzzahlige Lösungen für diese beiden Gleichungen sind somit:

					\begin{math}
						a = 15 \cdot n 
					\end{math} \begin{math}
						n \in \mathbb{Z}
					\end{math}

		            \begin{math}
			            b = 20 - 19n
		            \end{math}
					
					\begin{math}
			            c = 80 + 4n
		            \end{math}
	      \end{enumerate}
	\item Eine Gruppe von Menschen heißt \textit{halbzerstritten},
	      wenn für zwei beliebig herausgegriffene Gruppenmitglieder \begin{math}
		      a
	      \end{math} und \begin{math}
		      b
	      \end{math} immer gilt:
	      \textit{Entweder} redet \begin{math}
		      a
	      \end{math} mit \begin{math}
		      b
	      \end{math} oder \begin{math}
		      b
	      \end{math} redet mit \begin{math}
		      a
	      \end{math}.
	      Es gibt also weder Paare, bei denen Kommunikation in beide Richtungen möglich ist,
	      noch solche, bei denen Kommunikation in keine Richtung funktioniert.

	      Um in einer halbzerstrittenen Gruppe Nachrichten weiterzuleiten, wäre es hilfreich,
	      wenn man eine Kontaktperson \begin{math}
		      x
	      \end{math} hat, die mit jeder andereren Person über höchstens eine Zwischenstation redet -
	      also für jedes \begin{math}
		      a \neq x
	      \end{math} gilt,
	      dass \begin{math}
		      x
	      \end{math} mit \begin{math}
		      a
	      \end{math} redet oder es zumindest ein \begin{math}
		      b
	      \end{math} gibt, so dass \begin{math}
		      x
	      \end{math} mit \begin{math}
		      b
	      \end{math} und \begin{math}
		      b
	      \end{math} mit \begin{math}
		      a
	      \end{math} redet.

	      Gibt es eine solche Kontaktperson in jeder halbzerstrittenen Gruppe?

	      Lösung:
	      \begin{enumerate}
		      \item
	      \end{enumerate}
\end{enumerate}
\end{document}