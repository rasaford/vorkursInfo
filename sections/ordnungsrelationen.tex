\documentclass[../main.tex]{subfiles}
    \usepackage{amsmath}
    \usepackage{amsfonts}
    \usepackage{listings}
    
\begin{document}
\begin{enumerate}
	\item  Was für partielle Ordnungen und was für totale Ordnungen gibt es
	      auf zweielementigen Mengen \( \{x, y\} \)?

	      Lösung:
	      \begin{enumerate}
		      \item
	      \end{enumerate}
	\item Führe das im Beweis verwendete Sortierverfahren für die Menge \( A = \{ d,b,c,a,f,e \} \)
	      mit der alphabetischen Sortierung durch. Verwende als Pivotelement \( z \) immer
	      das vorderste Element (am Anfang also \( d \)) und lasse die Reihenfolge der
	      Elemente in \( A_1 \) und \( A_3 \) so, wie sie in \( A \) waren (also nicht beim Zerlegen aus
	      Versehen sortieren)!

	      Lösung:
	      \begin{enumerate}
		      \item
	      \end{enumerate}
	\item Beweise, dass die \( A_i \) aus dem Beweis in der Vorlesung paarweise disjunkt sind.

	      Lösung:
	      \begin{enumerate}
		      \item
	      \end{enumerate}
	\item Eine schöne Darstellung einer Relation \( m \) auf einer endlichen (und nicht zu
	      großen) Menge \( A \) ist mittels einer Tabelle, bei der Zeilen und Spalten mit den
	      Elementen von \( A \) beschriftet sind (beides Mal dieselbe Reihenfolge wählen)
	      und in der wir genau dann in Zeile \( x \) und Spalte \( y \) ein Kreuz machen, wenn
	      \( xRy \) gilt.
	      \begin{itemize}
		      \item  Wie sieht die Tabelle für die Teilmengenrelation auf \( \mathcal{P}(\{ 1, 2 \}) \) aus?
		      \item Wie sieht man einer solchen Tabelle an, ob die Relation
		            \begin{itemize}
			            \item reflexiv
			            \item total und / oder
			            \item antisymmetrisch
		            \end{itemize}
		            ist?
		      \item Wie sieht die Tabelle einer Ordnung aus, wenn Zeilen- und Spaltenbeschriftungen
		            nach dieser Ordnung sortiert sind?
	      \end{itemize}

	      Lösung:
	      \begin{enumerate}
		      \item
	      \end{enumerate}
	\item  Für eine endliche Menge \( A \) sei jedes Element \( x \in A \) mit einer Rangziffer
	      \( r(x) \in \mathbb{N} \) versehen. Wir betrachten die Relation \( R \) auf \( A \) mit
	      \[ xRy: \Leftrightarrow r(x) \leq r(y) \]
	      \begin{itemize}
		      \item Zeige, dass \( R \) transitiv, reflexiv und total ist.
		      \item Unter welcher Bedingung an die Beschriftungen \( r(x) \) ist \( R \) antisymmetrisch,
		            also eine Ordnung?
	      \end{itemize}

	      Lösung:
	      \begin{enumerate}
		      \item
	      \end{enumerate}
	\item Wenn man für eine endliche Menge \( A \) mit \( n = |A| \) Elementen nur eine
	      partielle Ordnung hat, funktioniert das Sortieren nicht. (Warum nicht?)

	      Es gibt aber einen entsprechenden Satz, der besagt, dass man \( A \) mit seiner
	      partiellen Ordnung topologisch sortieren kann, d.h. es gibt immer (mindestens)
	      eine Numerierung \( A = \{ x_1, x_2, \dots, x_n \} \), für die gilt:
	      \[ \forall i, j \in \{ 1, \dots, n\} : x_iRx_j \Rightarrow r \leq j \]

	      Beweise mithilfe dieses Satzes folgende Aussage: Zu jeder partiellen Ordnung
	      \( R \) auf einer endlichen Menge \( A \) gibt es eine totale Ordnung \( R' \)
	      mit \( R \subseteq R' \). (In Worten:
	      Jede partielle Ordnung auf einer endlichen Menge kann ich ”durch
	      Hinzufugen von Pfeilen“ zu einer totalen Ordnung ausbauen.)

	      Lösung:
	      \begin{enumerate}
		      \item
	      \end{enumerate}
\end{enumerate}
\end{document}