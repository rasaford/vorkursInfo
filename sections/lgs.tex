\documentclass[../main.tex]{subfiles}
    \usepackage{amsmath}
    \usepackage{listings} 
    \usepackage{amsfonts}

\begin{document}
\begin{enumerate}
	\item Gegeben sei das linare Gleichungsystem
	      \[ \begin{array}{cccc}
			      -x_1 & +2 x_2 & = & 2 \\
			      2x_1 & -x_2   & = & 2 \\
		      \end{array} \]
	      \begin{enumerate}
		      \item Löse das System zunächst graphisch.
		      \item Eliminiere nun mittels der ersten Gleichung das \( x_1 \)
		            in der zweiten Gleichung
		      \item Löse das so geänderte System noch einmal graphisch.
		      \item Berechne schließlich aus dem geänderten System die Lösung.
		\end{enumerate} 
		
		Lösung:
		\begin{enumerate}
			\item 
		\end{enumerate}
	\item Löse das linare Gleichungsystem
	      \[ \begin{array}{cccccc}
			      2x_1  & +2x_2 & -x_3  & -2x_4 & = & -1 \\
			      4x_1  & +4x_2 & -3x_3 & -x_4  & = & 5  \\
			            & 3x_2  & +x_3  & -x_4  & = & 1  \\
			      -2x_1 & +4x_2 & +4x_3 & +2x_4 & = & -2 \\
		      \end{array} \]
		
		Lösung:
		\begin{enumerate}
			\item 
		\end{enumerate}
	\item Das Lösen eines LGS nach dieser Methode benötigt bei \( n \) Unbekannten etwa
	      \( n^3/3 \) Operationen (Additionen und Multiplikationen). Angenommen, unser
	      Rechner schafft \( 100 \) Millionen Operationen pro Sekunde — wie lange braucht
	      er dann fur ein LGS mit \( 10 \), mit \( 1000 \), mit \( 100000 \) Unbekannten?
		
		Lösung:
		\begin{enumerate}
			\item 
		\end{enumerate}
	\item Für eine Matrix \( A \in \mathbb{R}^{ r \times s } \) (d.h. \( r \) Zeilen und \( s \)
	      Spalten, Koeffizienten aus
	      \( \mathbb{R} \)) und einen Vektor \( b \in \mathbb{R}^s \) ist das Matrix-Vektor-Produkt
	      \( c = A \cdot b \in \mathbb{R}^r \) definiert, bei dem in Zeile \( i \) das Skalarprodukt
	      aus der Zeile \( i \) von \( A \) und dem Vektor \( b \) gebildet wird:
	      \[ c_i = \sum_{ k = 1 }^{s} a_{i,k} \cdot b_k \]

	      Berechne folgendes Matrix-Vektor-Produkt
	      \[  \begin{pmatrix}
			      2  & 2 & -1 & -2 \\
			      4  & 4 & -3 & -1 \\
			      0  & 3 & 1  & 1  \\
			      -2 & 4 & 4  & 2  \\
		      \end{pmatrix}
		      \cdot \begin{pmatrix}
			      1  \\
			      0  \\
			      -1 \\
			      2  \\
		      \end{pmatrix}  \]
	      und überprüfe die Ergebnisse aus der Aufgabe 11.2
		
		Lösung:
		\begin{enumerate}
			\item 
		\end{enumerate}
	\item für welche Werte von \( a \) ist folgendes LGS lösbar?
	      Was sind dann die Lösungen?
	      \[ \begin{array}{ccccc}
			      x_1   & +x_2  & +x_3  & = & 2 \\
			      x_1   & +4x_2 & +3x_3 & = & 4 \\
			      -2x_1 & -3x_2 & -x_3  & = & a \\
		      \end{array} \]
		
		Lösung:
		\begin{enumerate}
			\item 
		\end{enumerate}
\end{enumerate}
\end{document}