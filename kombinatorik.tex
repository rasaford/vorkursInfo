\documentclass[main.tex]{subfiles}
    \usepackage{amsmath}
    \usepackage{listings}
    
\begin{document}
\begin{enumerate}
	\item Zeige:
	      \[ \binom{n}{n - k} = \binom{n}{k} \]

	      Lösung:
	      \begin{enumerate}
		      \item \( \binom{n}{n-k}
		            = \frac{n!}{(n-k)! (n - (n -k))!}
		            = \frac{n!}{(n - k)! k!} \)
	      \end{enumerate}
	\item Bei der Strich- Sternmalerei hätten wir auch mit Sternen anfangen können
	      und \( n - 1 \) davon durch Striche ersetzen. Stelle eine Formel für die Anzahl der Möglichkeiten hierfur auf!

	      Zeige, dass diese Formel dieselben Werte liefert wie
	      \[ \binom{ n + k - 1}{ k} \]

	      (na hoffentlich!)

	      Lösung:
	      \begin{enumerate}
		      \item \( \binom{n + k -1}{n-1}
		            = \binom{n + k -1}{(n + k - 1)- (n-1)}
		            = \binom{n + k -1}{k}\)
	      \end{enumerate}
	\item Zeige:
	      \[ \binom{n}{k} + \binom{n}{ k + 1} = \binom{n + 1}{k + 1} \]
	      und gib damit ein Schema zum Berechnen der Binomialkoeffizienten nur
	      mittels von Additionen an (zum Rechnen von Hand sehr praktisch!). Wo
	      haben wir dieses Schema schon mal gesehen?

	      Lösung:
	      \begin{enumerate}
		      \item \( \binom{n}{k} + \binom{n}{k + 1}
		            = \frac{n!}{k! (n-k)!} + \frac{n!}{(k+1)! (n - (k +1))!} \)

		            \( =  \frac{n!(k+1)}{(k+1)! (n-k)!} + \frac{n! (n - k)}{(k+1)! (n - k)!} \)

		            \( =  \frac{n!(k+1) + n!(n-k)}{(k+1)! (n - k)!} \)

		            \( =  \frac{n!(k+1 + n-k)}{(k+1)! (n - k)!} \)

		            \( =  \frac{n!(n + 1)}{(k+1)! (n - k)!} \)

		            \( =  \frac{(n + 1)!}{(k+1)! (n - k)!} \)

		            \( = \binom{n+1}{k+1} \)
	      \end{enumerate}
	\item  Gib die Koeffizienten des Polynoms \( (1 + x)^n \) mittels Binomialkoeffizienten
	      an und zeige durch geschickes Verwenden dieser Formel
	      \[ \sum_{ k= 0}^{n} \binom{n}{k} = 2^n, n > 0, \sum_{k = 0}^{n} (-1)^k \binom{n}{k} = 0 \]

	      Lösung:
	      \[ (a + b)^n = \sum_{k=0}^{n} \binom{n}{k} a^{n-k} \cdot b^k \]
	      \begin{enumerate}
		      \item \( (1+1)^n = \sum_{k = 0}^{n} \binom{n}{k} 1^{n-k} \cdot 1^k
		            = \sum_{k = 0}^{n} \binom{n}{k} \)
		      \item \( (1-1)^n = \sum_{k = 0}^{n} \binom{n}{k} 1^{n-k} \cdot (-1)^k
		            = \sum_{k = 0}^{n} \binom{n}{k} (-1)^k \)

	      \end{enumerate}
	\item Wie viele verschiedene Möglichkeiten zu tippen hat man beim klassischen
	      Lotto ” \( 6 \) aus \( 49 \)" (Berechnen mit Taschenrechner oder per Hand mit Runden
	      auf zwei gültige Ziffern)?

	      Lösung:
	      \begin{enumerate}
		      \item \( \binom{49}{6} = 13983816 \)
	      \end{enumerate}
	\item Wie viele verschiedene Möglichkeiten gibt es beim Fußball-Toto?
	      (13 Spiele sind zu tippen, jeweils Heimsieg, Heimniederlage oder Unentschieden –
	      wieder Taschenrechner verwenden oder runden).

	      Lösung:
	      \begin{enumerate}
		      \item \( 3^{13} = 1594323 \)
	      \end{enumerate}
	\item Wieviele mögliche Ergebnisse gibt es beim Würfeln mit \( n \) nicht unterscheidbaren
	      Würfeln (Ergebnis sind dabei die Punkte der einzelnen Würfel,
	      also wäre z.B. bei fünf Würfeln \( 2,3,5,6,6 \) ein mögliches Ergebnis)?

	      Lösung:
	      \begin{enumerate}
		      \item \( 6^n \)
	      \end{enumerate}
	\item Wie viele verschiedene ”Full House“ gibt es beim Poker?
	      (\( 52 \) Karten, vier Farben mit je \( 2,3,4, \dots, 10 \), Bube, Dame, König, As)

	      Lösung:
	      \begin{enumerate}
		      \item Beim Full House muss zuerst einen Symbol ausgewählt werden.
		            Für dieses gibt es \( \binom{13}{1} \) Möglichkeiten. Von diesem müssen dann
		            \( 3 \) der \( 4 \) Karten ohne Zurücklegen gezogen werden \( \binom{4}{3} \).
		            Dann wird ein weiteres Symbol, welches ungleich dem vorherigen sein muss gezogen.
		            Dies besitzt \( \binom{12}{1} \) Möglichkeiten. Von diesem werden dann \( 2 \) Karten benötigt.

		            Somit ergibt sich:
		            \[ \binom{13}{1} \cdot \binom{4}{3} \cdot \binom{12}{1} \cdot \binom{4}{2}
			            = 3744 \]
	      \end{enumerate}
	\item Wie viele Anagramme (Wörter, die aus denselben Buchstaben bestehen – es
	      geht nur um die möglichen Buchstabenvertauschungen, aussprechen können
	      muss man die Anagramme nicht; Groß- und Kleinbuchstaben werden nicht
	      unterschieden) gibt es von dem Wort \textit{Muh}? Wie viele Anagramme gibt es
	      von \textit{Atlantis} und wie viele von \textit{Mississippi}?

		Lösung:
		
		Die Anzahl der möglichen Permutationen einer Menge beträgt \( n! \). 
		Wenn aber \( k \) gleiche Elemente in dieser enthalten sind, beträgt die 
		Zahl der unterschiedlichen Permutationen \( \frac{n!}{k!} \), denn 
		\( k! \) Permutationen sind gleich. 
	      \begin{enumerate}
		      \item Muh: \( 3! \)
		      \item Atlantis: \( \frac{8!}{4!} = 8 \cdot 7 \cdot 6 \cdot 5 = 1680 \)
		      \item Mississippi: \( \frac{11!}{10!} = 11 \)
	      \end{enumerate}
\end{enumerate}
\end{document}
