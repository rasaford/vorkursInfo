\documentclass[main.tex]{subfiles}
    \usepackage{amsmath}
    \usepackage{listings}
    
\begin{document}
\begin{enumerate}
	\item Zeige:
	      \[ \binom{n}{n - k} = \binom{n}{k} \]

	      Lösung:
	      \begin{enumerate}
		      \item
	      \end{enumerate}
	\item Bei der Strich- Sternmalerei hätten wir auch mit Sternen anfangen können
	      und \( n - 1 \) davon durch Striche ersetzen. Stelle eine Formel für die Anzahl der Möglichkeiten hierfur auf!

	      Zeige, dass diese Formel dieselben Werte liefert wie
	      \[ \binom{ n + k - 1}{ k} \]

	      (na hoffentlich!)

	      Lösung:
	      \begin{enumerate}
		      \item
	      \end{enumerate}
	\item Zeige:
	      \[ \binom{n}{k} + \binom{n}{ k + 1} = \binom{n + 1}{k + 1} \]
	      und gib damit ein Schema zum Berechnen der Binomialkoeffizienten nur
	      mittels von Additionen an (zum Rechnen von Hand sehr praktisch!). Wo
	      haben wir dieses Schema schon mal gesehen?

	      Lösung:
	      \begin{enumerate}
		      \item
	      \end{enumerate}
	\item  Gib die Koeffizienten des Polynoms \( (1 + x)^n \) mittels Binomialkoeffizienten
	      an und zeige durch geschickes Verwenden dieser Formel
	      \[ \sum_{ k= 0}^{n} \binom{n}{k} = 2^n, n > 0, \sum_{k = 0}^{n} (-1)^k \binom{n}{k} = 0 \]

	      Lösung:
	      \begin{enumerate}
		      \item
	      \end{enumerate}
	\item Wie viele verschiedene Möglichkeiten zu tippen hat man beim klassischen
	      Lotto ” \( 6 \) aus \( 49 \)" (Berechnen mit Taschenrechner oder per Hand mit Runden
	      auf zwei gültige Ziffern)?

	      Lösung:
	      \begin{enumerate}
		      \item
	      \end{enumerate}
	\item Wie viele verschiedene Möglichkeiten gibt es beim Fußball-Toto?
	      (13 Spiele sind zu tippen, jeweils Heimsieg, Heimniederlage oder Unentschieden –
	      wieder Taschenrechner verwenden oder runden).

	      Lösung:
	      \begin{enumerate}
		      \item
	      \end{enumerate}
	\item Wieviele mögliche Ergebnisse gibt es beim Würfeln mit \( n \) nicht unterscheidbaren
	      Würfeln (Ergebnis sind dabei die Punkte der einzelnen Würfel,
	      also wäre z.B. bei fünf Würfeln \( 2,3,5,6,6 \) ein mögliches Ergebnis)?

	      Lösung:
	      \begin{enumerate}
		      \item
	      \end{enumerate}
	\item Wie viele verschiedene ”Full House“ gibt es beim Poker?
	      (\( 52 \) Karten, vier Farben mit je \( 2,3,4, \dots, 10 \), Bube, Dame, König, As)

	      Lösung:
	      \begin{enumerate}
		      \item
	      \end{enumerate}
	\item Wie viele Anagramme (Wörter, die aus denselben Buchstaben bestehen – es
	      geht nur um die möglichen Buchstabenvertauschungen, aussprechen können
	      muss man die Anagramme nicht; Groß- und Kleinbuchstaben werden nicht
	      unterschieden) gibt es von dem Wort \textit{Muh}? Wie viele Anagramme gibt es
	      von \textit{Atlantis} und wie viele von \textit{Mississippi}?

	      Lösung:
	      \begin{enumerate}
		      \item
	      \end{enumerate}
\end{enumerate}
\end{document}
