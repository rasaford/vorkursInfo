\documentclass[main.tex]{subfiles}
    \usepackage{amsmath}
    \usepackage{amsfonts}
    \usepackage{listings}

\begin{document}
\begin{enumerate}
	\item Ein Rechteck \( R \) mit Seitenlägen \( a \) und \( b \)
	      habe eine Fläche von \( 10 cm^2 \). Drücke den Umfang \( U \) von \( R \) als Funktion von \( b \) aus.

	      Lösung:
	      \begin{enumerate}
		      \item Fläche des Rechtecks:
		            \( a \cdot b = 10 cm^2 | \div b\)

		            \( a = \frac{10 cm^2}{b} \)

		            Umfang des Rechtecks:
		            \( u = 2(a + b)
		            = 2 ( b + \frac{10 cm^2}{b} ) \)

		            \[ f: \mathbb{R} \rightarrow \mathbb{R}; b \mapsto 2 \cdot \Big( b + \frac{10 cm^2}{b} \Big)  \]
	      \end{enumerate}
	\item  Gib (z.B. als Tabelle) die Und-Verknüpfung als Abbildung \( \mathbb{B} \times \mathbb{B} \rightarrow \mathbb{B} \) an,
	      wobei \( \mathbb{B} := \{ ture, false \} \) die Menge der Wahrheitswerte sei.
	      Ist die Abbildung injektiv und/oder surjektiv?

	      Lösung:
	      \begin{enumerate}
		      \item \[
			            \begin{array}{c|c|c}
				            \land & true  & false \\
				            \hline
				            true  & true  & false \\
				            \hline
				            false & false & false \\
			            \end{array}
		            \]

		            Die und-Verknüpfung ist nicht injektiv, da:

		            \( f^{-1}(false) = \{\{false, true\}, \{ture, false\}, \{false, false\}\} \)

		            Sie ist aber surjektiv.
	      \end{enumerate}
	\item Existieren Abbildungen, die weder surjektiv noch injektiv sind?
	      Gib gegebenenfalls solche Abbildungen an und veranschauliche sie anhand einer Skizze.

	      Lösung:
	      \begin{enumerate}
		      \item \( f: \mathbb{R} \rightarrow \mathbb{R}; x \mapsto \sqrt{x} \)

		            Nicht injektiv, da:

		            \( f^{-1}(5) = \{25, -25\} \)

		            Nicht surjektiv, da:

		            \( f^{-1}(-1) = \emptyset \)
	      \end{enumerate}
	\item Prüfe die folgenden Funktionen auf Injektivität und Surjektivität:
	      \begin{enumerate}
		      \item \( f: \mathbb{Z} \rightarrow \mathbb{N}_0 ; x \mapsto x^2 \)
		      \item \( f: \mathbb{N}_0 \rightarrow \mathbb{N}_0 ; x \mapsto x^2 \)
		      \item \( f: \mathbb{N} \rightarrow \mathbb{N}; 1 \mapsto 1, x \mapsto x - 1 \) für \( x > 1 \)
		      \item \( f: \mathbb{Z} \rightarrow \mathbb{Z}; x \mapsto x - 1 \)
	      \end{enumerate}

	      Lösung:
	      \begin{enumerate}
		      \item \( f: \mathbb{Z} \rightarrow \mathbb{N}_0 ; x \mapsto x^2 \)

		            nicht injektiv: \( f^{-1}(25) = \{-5, 5\} \)

		            nicht surjektiv: \( f^{-1}(2) = \sqrt{2} \notin \mathbb{Z} \)
		      \item \( f: \mathbb{N}_0 \rightarrow \mathbb{N}_0 ; x \mapsto x^2 \)

		            injektiv

		            nicht surjektiv: \( f^{-1}(2) = \sqrt{2} \notin \mathbb{Z} \)
		      \item \( f: \mathbb{N} \rightarrow \mathbb{N}; 1 \mapsto 1, x \mapsto x - 1 \) für \( x > 1 \)

		            nicht injektiv: \( f^{-1}(1) = \{1,2\} \)

		            surjektiv
		      \item \( f: \mathbb{Z} \rightarrow \mathbb{Z}; x \mapsto x - 1 \)

		            injektiv

		            surjektiv
	      \end{enumerate}
	\item Gib jeweils zwei Funktionen von \( \mathbb{N} \) nach \( \mathbb{N} \) an, die
	      \begin{enumerate}
		      \item injektiv, aber nicht surjektiv
		      \item bijektiv sind.
	      \end{enumerate}

	      Lösung:
	      \begin{enumerate}
		      \item \( f:\mathbb{N} \rightarrow \mathbb{N}; x \mapsto x + 1 \)

		            \( f:\mathbb{N} \rightarrow \mathbb{N}; x \mapsto 2x \)

		      \item \( f:\mathbb{N} \rightarrow \mathbb{N}; x \mapsto x \)

		            \( f:\mathbb{N} \rightarrow \mathbb{N}; x \mapsto x + 1 \) für \( x \) gerade,
		            \( x - 1 \) für \( x \) ungerade

	      \end{enumerate}
	\item Gib eine bijektive Abbildung \( f: \mathbb{N}_0 \rightarrow Z \) an.

	      Lösung:
	      \begin{enumerate}
		      \item \( f: \mathbb{N}_0 \rightarrow \mathbb{Z}; 0 \mapsto 0, \)

		            \hspace{52pt}\( x \mapsto \frac{x}{2} \) für \( x \) gerade,

		            \hspace{52pt}\( x \mapsto - \frac{x + 1}{2} \) für \( x \) ungerade
	      \end{enumerate}
	\item Ordnet man jedem auftretenden Funktionswert \( y \) einer Abbildung  \( f: A \rightarrow B \)
	      die Elemente seiner Urbildmenge \( f^{-1}(y) \) zu, so erhält man eine Relation.
	      Unter welchen Bedingungen ist diese Umkehrrelation eine Abbildung?

	      Die in diesem Fall definierte Abbildung \( B \rightarrow A \) heißt \textit{Umkehrabbildung} und
	      wird — etwas leichtsinnig — auch mit \( f^{-1} \) bezeichnet.

	      Lösung:
	      \begin{enumerate}
		      \item
	      \end{enumerate}
	\item Die Bijektivität kann man gut einsetzen, um zu entscheiden, ob zwei Mengen
	      gleich viele Elemente haben – dies ist genau dann der Fall, wenn es eine
	      bijektive Abbildung zwischen den beiden Mengen gibt.
	      \begin{itemize}
		      \item Demonstriere mit ein paar einfachen Beispielen, dass diese Definition
		            für ”kleine“ Mengen gut funktioniert.
	      \end{itemize}
	      Im Gegensatz zum intuitiven ”Zählen der Elemente“ lässt sich das Kriterium
	      Bijektivität auch auf unendliche Mengen übertragen:
	      \begin{itemize}
		      \item Vergleiche mit diesem Kriterium die Menge der geraden Zahlen mit der
		            der ungeraden Zahlen.
		      \item Gibt es mehr Zahlen die durch 2 teilbar sind, oder mehr, die durch 3
		            teilbar sind?
	      \end{itemize}
	      Mehr Erstaunliches über Bijektionen auf unendlichen Mengen findet man unter dem Stichwort
	      ”Hilberts Hotel“, z.B. in der Wikipedia.

	      Lösung:
	      \begin{enumerate}
		      \item
	      \end{enumerate}
	\item  Zusatzaufgabe: Zeige, dass auch für unendliche Mengen \( A \) gilt: \( |A| \neg |\mathcal{P}(a)| \),
	      d.h., es gibt keine Bijektion zwischen \( A \) und \( \mathcal{P}(A) \).
	      Tipp: zu einer Bijektion \( f: A \rightarrow \mathcal{P}(a) \) könnte man sich die Menge aller
	      \( x \in A \) ansehen, für die \( x \notin f(x) \) gilt\dots

	      Lösung:
	      \begin{enumerate}
		      \item
	      \end{enumerate}
\end{enumerate}
\end{document}