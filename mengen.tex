\documentclass[main.tex]{subfiles}
    \usepackage{amsmath}
    \usepackage{amsfonts}
    \usepackage{listings}
    \usepackage{multicol}
    
\begin{document}
\begin{enumerate}
	\item Gegeben sind die folgenden Teilmengen \(A = \{ 1, 3, 5, 7, 9 \}, B = \{2, 4, 6, 8, 10 \} \) und
	      \(D = \{ 5,6,7,8,9,10\} \).

	      Gib die folgenden Mengen an:
	      \begin{multicols}{3}
		      \begin{enumerate}
			      \item \(A \cup B \)
			      \item \(A \cap B \)
			      \item \(A \setminus B \)
			      \item \(A \setminus D \)
			      \item \(B \setminus D \)
			      \item \(D \setminus A \)
			      \item \(D \setminus B \)
			      \item \(D \setminus(A \cup B) \)
			      \item \(D \setminus(A \cap B) \)
		      \end{enumerate}
	      \end{multicols}

	      Lösung:
	      \begin{enumerate}
		      \item
	      \end{enumerate}
	\item Wie viele Elemente enthält die Potenzmenge \( \mathcal{P}(A) \) einer (endlichen)
	      Menge \(A \) mit \( |A| = n \)? Schreibe z.B. alle Teilmengen von \( \{1,2\} \) oder
	      \( \{1,2,3\} \) auf, und versuche eine Regelmäßigkeit zu erkennen.
	      Wie könnte man die Regelmäßigkeit allgemein beweisen?
	      Zeige dass für endliche Mengen stets \( |A| < |\mathcal{P}(A)| \) gilt.

	      Lösung:
	      \begin{enumerate}
		      \item
	      \end{enumerate}
	\item Bestimme die folgenden Mächtigkeiten:
	      \begin{multicols}{2}
		      \begin{enumerate}
			      \item \( |\{1, 4, ,6 \}| \)
			      \item \( |\emptyset| \)
			      \item \( |\{ \emptyset \}| \)
			      \item \( |\{ \emptyset, \{ 1, 2 ,3 \} \}| \)
		      \end{enumerate}
	      \end{multicols}
	      Lösung:
	      \begin{enumerate}
		      \item
	      \end{enumerate}
	\item Zeichne Punktmengen \( A, B \) und \( C \), die die folgenden vier Bedingungen
	      zugleich erfüllen:
	      \begin{multicols}{2}
		      \begin{enumerate}
			      \item \( A \cap B \cap C = \emptyset \)
			      \item \( A \cap B \neq \emptyset \)
			      \item \( B \cap C \neq \emptyset \)
			      \item \( A \cap C \neq \emptyset \)
		      \end{enumerate}
	      \end{multicols}
	      Gib daraufhin Zahlenmengen möglichst kleiner Mächtigkeit an, die diese
	      Bedingungen erfüllen.

	      Lösung:
	      \begin{enumerate}
		      \item
	      \end{enumerate}
	\item \( A, B \) und \( C \) seien Teilmengen einer Grundmenge \( G \).
	      Beweise von den folgenden Aussagen die wahren und gib für die falschen jeweils ein Gegenbeispiel an.
	      \begin{enumerate}
		      \item Wenn \( A \cup B = A \cup C \), dann ist \( B = C \)
		      \item Wenn \( A \setminus B = A \), dann ist \( B = C \)
		      \item Wenn \( B = \emptyset \), dann ist \(  A \setminus B  = A \)
		      \item \( A \setminus B  \) und \( B \setminus C \) sind immer disjunkt
		            (d.h. die Schnittmenge ist leer).
	      \end{enumerate}

	      Lösung:
	      \begin{enumerate}
		      \item
	      \end{enumerate}
	\item Beweise, dass zwei Mengen \( A \) und \( B \) gleich sind,
	      wenn sie wechselseitig Teilmengen voneinander sind (und auch nur dann),
	      also:
	      \[ A = B \Leftrightarrow A \subseteq B \land B \subseteq A \]

	      Lösung:
	      \begin{enumerate}
		      \item
	      \end{enumerate}
	\item Die \( 30 \) Schüler einer Klasse schrieben in den drei Fächern
	      Deutsch, Englisch und Mathematik Prüfungsarbeiten mit folgendem Ergebnis:
	      In Deutsch bestanden \( 22 \), in Englisch bestanden \( 17 \)
	      und in Mathematik bestanden \( 22 \) Schüler.
	      \( 4 \) bestanden weder Deutsch noch Englisch,
	      \( 3 \) bestanden weder Deutsch noch Mathematik,
	      \( 5 \) bestanden weder Englisch noch Mathematik.
	      \( 1 \) Schüler schaffte keine der drei Prüfungen.

	      Wie viele Schüler bestanden die Prüfung in allen drei Fächern?Aussagen

	      Hinweis: zeichne die Mengen!

	      Lösung:
	      \begin{enumerate}
		      \item
	      \end{enumerate}
	\item Mit der Schreibweise
	      \[ \bigcup_{k = 1}^{n} A_k := A_1 \cup A_2 \cup \dots \cup A_n \]
	      kann man bequem auch kompliziertere Mengen formulieren, insbesondere dann, wenn man
	      erlaubt, dass auch unendlich viele Mengen vereinigt werden dürfen:
	      \[ \bigcup_{k = 1}^{\infty} A_k := A_1 \cup A_2 \cup \dots \cup A_n \cup \dots \]
	      Ein Element ist in dieser Vereinigungsmenge enthalten, wenn es in einer der Mengen \( A_k \) enthalten ist.
	      Überlege Dir, wie man zum Beispiel die Menge der Primzahlen hinschreiben könnte
	      (Tipp: formuliere dazu z.B. die Menge \( V_2 \) der Vielfachen von \( 2 \), etc.).

	      Lösung:
	      \begin{enumerate}
		      \item
	      \end{enumerate}
\end{enumerate}
\end{document}