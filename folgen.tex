\documentclass[main.tex]{subfiles}
    \usepackage{amsmath}
    \usepackage{listings}
	\usepackage{amsfonts}
	\usepackage{multicol}
    
\begin{document}
\begin{enumerate}
	\item Berechne die ersten Partialsummen der Reihe mit Gliedern \( s_n = \frac{9}{10^n} \) und
	      bestimme deren kleinste obere Schranke.

	      Lösung:
	      \begin{enumerate}
		      \item Die Partialsumme der Reihenglieder ist:
		            \[ \sum_{i = 1}^{n} \frac{9}{10^i}  = \frac{10^n - 1}{10^n}\]
		            \( \sum_{i = 1}^{1} \frac{9}{10^i} = \frac{9}{10} \)

		            \( \sum_{i = 1}^{2} \frac{9}{10^i} =\frac{9}{10} + \frac{9}{100} = \frac{99}{100} \)

		            \( \sum_{i = 1}^{3} \frac{9}{10^i} = \frac{9}{10} + \frac{9}{100} + \frac{9}{1000} = \frac{999}{1000} \)

		            Im Grenzwert für \( n \to \infty \) konvergiert diese Summe gegen \( 1 \)

		            \[ \lim_{n \to \infty} \frac{10^n - 1}{10^n}
			            = \lim_{n \to \infty} \Big( \frac{10^n}{10^n} - \frac{1}{10^n} \Big)
			            = \lim_{n \to \infty} \Big( 1 - \frac{1}{10^n} \Big) = 1\]
	      \end{enumerate}
	\item Gib eine Bedingung an die Reihenglieder \( s_k \) an, wann eine Reihe \( f_n = \sum s_k \)
	      streng monoton wachsend bzw. fallend ist (im Fall einer Reihe beziehen sich
	      Attribute wie ”monoton wachsend“ auf die Partialsummen, nicht auf die
	      Reihenglieder).

	      Lösung:
	      \begin{enumerate}
		      \item Es müssen alle \( s_k > 0 \) sein damit \( f_n \) streng monoton wachsend ist.
		            Wenn alle \( s_k < 0 \) ist die Folge streng monoton fallend.
	      \end{enumerate}
	\item Zeige für \( 0 < x < 1  \) ist die \textit{geometrische Reihe}
	      \[ f_n := \sum_{k = 0}^{n} x^k \]

	      beschränkt und monoton, mithin konvergent. Was ist der Grenzwert der Reihe?
	      (Tipp: Die Summe kennen wir schon\dots )

	      Lösung:
	      \begin{enumerate}
		      \item Jeder Term der Partialialsumme ist \(  > 0 \). Somit ist die Folge der Partialsummen
		            streng monoton wachsend. Sie konvergiert gegen :
		            \[ \sum_{i = 0}^{n} x^i = \frac{x^{n+1} - 1}{x - 1} \]

		      \item Konvergenz von \( f_n \)
	      \end{enumerate}
	\item Zeige, dass die harmonische Reihe mit Reihengliedern \( s_n := \frac{1}{n} \) keine obere
	      Schranke besitzt (die Partialsummen also beliebig groß werden).

	      Lösung:
	      \begin{enumerate}
		      \item
		            \( \sum_{i = 0}^{\infty} \frac{1}{i} \)

		            \( = 1 + \frac{1}{2} + \frac{1}{3} + \frac{1}{4} + \frac{1}{5} + \frac{1}{6}  + \frac{1}{7}  + \frac{1}{8}  + \frac{1}{9} +  \dots \)

		            \( > 1 + \frac{1}{2} + \frac{1}{4} + \frac{1}{4} + \frac{1}{8} + \frac{1}{8}  + \frac{1}{8} + \frac{1}{8}  + \frac{1}{16}+ \dots \)

		            \( = 1 + \frac{1}{2} + \frac{1}{2} + \frac{1}{2} + \dots \)

		            \( = 1 + \sum_{i = 0}^{\infty} \frac{1}{2} \Rightarrow \) nicht konvergent
	      \end{enumerate}
	\item Zeige, dass eine Folge höchstens einen Grenzwert besitzen kann!
	      Nimm dazu an, es gebe \( y_1 \neq y_2 \), die beide die Grenzwertbedingung erfüllen und
	      führe das zum Widerspruch durch Angabe eines \( \epsilon \), mit dem die weitere Bedingung
	      unmöglich erfüllt sein kann.

	      Lösung:
	      \begin{enumerate}
		      \item Widerspruchsbeweis:

		            Annahme: Die Folge \( f_n \) hat die Grenzwerte \( y_1, y_2 \) mit \( y_1 \neq y_2 \)

		            Nach der Definition des Grenzwerts einer Folge muss somit gelten:
		            \[ \forall \epsilon > 0: \exists n_0(\epsilon) \in \mathbb{N}: n \geq n_0(\epsilon) \Rightarrow |f_n - y_1| < \epsilon \]
		            \[ \forall \epsilon > 0: \exists n_1(\epsilon) \in \mathbb{N}: n \geq n_0(\epsilon) \Rightarrow |f_n - y_2| < \epsilon \]

		            Da die Grenzwerte unterschiedlich sind, wird \( \epsilon \) (Radius um den Grenzwert, in dem unendlich viele Punkte liegen müssen)
		            so gewählt, dass es kleiner als die hälfte der Distanz von \( y_1, y_2 \) auf dem Zahlenstrahl.
		            \[ \epsilon < \frac{|y_1 - y_2|}{2} \]
		            Somit müssen die Mengen der Punkte, die in den \( \epsilon \)- Umgebungen von \( y_1, y_2 \) liegen disjunkt sein.

		            Es wird nun \( N := \max\{n_0, n_1\} \) gewählt. Somit sind alle Glieder der Folge \( f_n \)  für beide Grenzwerte innerhalb
		            der gegebenen \( \epsilon  \)-Umgebung.
		            \[ |f_n - y_1| + |f_n - y_2| < \epsilon + \epsilon = 2\epsilon \]

		            \( 2\epsilon =  2 \cdot \frac{|y_1 - y_2|}{2} = |y_1 - y_2|\)

		            \( |y_1 - f_n + f_n - y_2| \)

		            \( |(y_1 - f_n) + (f_n - y_2)| \)

		            \( \stackrel{\text{Dreiecksungleichung}}{\leq} |y_1 -f_n| + |f_n - y_2| \)

		            \( |f_n - y_1| + |f_n - y_2| \)

		            Somit ergibt sich der folgende Widerspruch. Die Annahme muss also falsch sein.
					\[ 2 \epsilon \leq  |f_n - y_1| + |f_n - y_2| < 2 \epsilon \]
					
					\( \square \)
	      \end{enumerate}
	\item Ein Mann spaziert mit seinem Hund von seinem Haus zu einer Kneipe. Die
	      Entfernung zwischen Haus und Kneipe sei \( s \). Der Mann gehe dabei mit der
	      Geschwindigkeit \( v \). Dies ist dem Hund jedoch zu langweilig. Er läuft deswegen
	      doppelt so schnell zwischen der Kneipe und seinem Herrchen hin und her.
	      Das heißt, er startet am Haus zusammen mit seinem Herrchen, dreht um,
	      sobald er das Ziel erreicht, stoppt, wenn er wieder auf sein Herrchen trifft,
	      läuft dann wieder zur Kneipe,\dots

	      Berechne, welchen Weg der Hund zurücklegt, bis Herrchen und Hund gemeinsam die Kneipe erreichen.
	      Es gibt einen so genannten ”Mathematiker-Weg“ und einen so genannten
	      ”Physiker-Weg“. Versuche, beide zu finden.

	      Lösung:
	      \begin{enumerate}
		      \item Mathematiker Weg:

		            Da der Hund doppelt so schnell läuft wie der Mann und umdreht, sobald sich die Beiden treffen, legt er
		            folgende Strecke zurück:

		            \( 1 + \frac{1}{2} + \frac{1}{4} + \frac{1}{8} + \dots = \sum_{i = 0}^{\infty} \frac{1}{2^i}\)

		            \( = \frac{1}{1 - \frac{1}{2}} = 2  \)
		      \item Physiker Weg:

		            Da der Hund sich in der gleichen Zeit doppelt so schnell bewegt, wie der Mann, legt der die doppelte Strecke zurück.
	      \end{enumerate}
	\item  In Aufgabe \( 2.7 \) haben wir die Länge der Kochkurve zu berechnet, was uns auf
	      eine Folge \( (l_n) \) von Zahlen führt. Ebenso kann man die Fläche \( (a_n) \) unter der
	      Kochkurve berechnen, indem man die Flächen der Dreiecke aufaddiert, die
	      Schritt fur Schritt auf die Kurve ”draufgesetzt“ werden.
	      Sind diese Folgen jeweils beschränkt und/oder monoton?

	      Lösung:
	      \begin{enumerate}
		      \item
	      \end{enumerate}
	\item Gib Folgen \( (a_n) \), \( (b_n) \) an mit \( \lim_{n \to \infty}(a_n) = \infty \),
	      \( \lim_{n \to \infty}(b_n) = \infty \) sowie:
	      \begin{enumerate}
		      \item \( \lim(a_n - b_n) = 0 \)
		      \item \( \lim(a_n - b_n) = + \infty \)
		      \item \( \lim( \frac{a_n}{b_n}) = 0 \)
	      \end{enumerate}

	      Lösung:
	      \begin{enumerate}
		      \item \( a_n = n, b_n = n \)

		            \( \lim_{n \to \infty} (n - n) = 0 \)
		      \item \( a_n = 2n, b_n = n \)

		            \( \lim_{n \to \infty} (2n - n) = \lim_{n \to \infty} n = \infty \)
		      \item \( a_n = n, b_n = n^2 \)

		            \( \lim_{n \to \infty} \frac{n}{n^2} = \lim_{n \to \infty} \frac{1}{n} = 0  \)
	      \end{enumerate}
	\item \( O \)- Notation: zeig, dass gilt:
	      \begin{enumerate}
		      \item \( 6n^4  \in O(3n^4) \)
		      \item \( 16n^3 \in O(2^n) \)
		      \item \( n^2 \notin O(n) \)
	      \end{enumerate}
	      Gib im Fall von "`\( \in \)"' ein entsprechendes \( n_0 \) an, so dass Bedingung
	      (8.1) erfüllt ist.

	      Lösung:
	      \begin{enumerate}
		      \item
	      \end{enumerate}
	\item Bestmme alle Häufungspunkte und den größten Häufungspunkt von
	      \[ a_n = \sin \Big( \frac{\pi}{4}n \Big) \]
	      (Benutze dabei die Gleichheit \( \sin( \pi / 4 ) = \sqrt{2} / 2 \)).

	      Lösung:
	      \begin{enumerate}
		      \item
	      \end{enumerate}
	\item Prüfe jeweils auf Konvergenz und bestimme ggf. den Grenzwert.

	      Hinweis: Wenn \( \lim_{n \to \infty} a_n = \alpha \) und
	      \( \lim_{n \to \infty} b_n = \beta \) dann ist
	      \( \lim_{n \to \infty} (a_n + b_n) = \alpha + \beta \).

	      Ebenfalls gilt: \( \lim_{n \to \infty} (a_n \cdot b_n) = \alpha \cdot \beta \) und
	      - wenn \( \beta \)  und alle \( b_n \neq 0 \) sind -
	      \( \lim_{n \to \infty }(a_n / b_n) = \alpha / \beta \)
	      \begin{enumerate}
		      \item \( a_n = \frac{ 3n +2(-1)^n }{ n } \)
		      \item \( b_n = \frac{ nx^n }{ nx^n + 1 } \)
		            \( x \in \mathbb{R}, x > 1 \)
	      \end{enumerate}

	      Lösung:
	      \begin{enumerate}
		      \item \( lim_{n \to \infty} a_n
		            = \lim_{n \to \infty} \frac{3n + 2(-1)^n}{n} \)

		            \( = \lim_{n \to \infty} \Big( \frac{3n}{n} + \frac{2(-1)^n}{n} \Big) \)

		            \( = \lim_{n \to \infty} \Big( 3 + \frac{2(-1)^n}{n} \Big) = 3 \)
		      \item \( \lim_{n \to \infty} b_n
		            = \lim_{n \to \infty} \frac{ nx^n }{ nx^n + 1 } \)

		            \( = \lim_{n \to \infty} \Big( \frac{ \frac{nx^n}{nx^n} }{ \frac{nx^n + 1}{ nx^n} }  \Big) \)

		            \( = \lim_{n \to \infty} \Big( \frac{1}{ 1 + \frac{1}{ nx^n} }  \Big) = 0 \)

	      \end{enumerate}
\end{enumerate}
\end{document}