\documentclass[main.tex]{subfiles}
    \usepackage{amsmath}
    \usepackage{listings}
    \usepackage{amsfonts}
    
\begin{document}
\begin{enumerate}
	\item Berechne die ersten Partialsummen der Reihe mit Gliedern \( s_n = \frac{9}{10^n} \) und
          bestimme deren kleinste obere Schranke. 
          
          Lösung:
          \begin{enumerate}
              \item 
          \end{enumerate}
	\item Gib eine Bedingung an die Reihenglieder \( s_k \) an, wann eine Reihe \( f_n = \sum s_k \)
	      streng monoton wachsend bzw. fallend ist (im Fall einer Reihe beziehen sich
	      Attribute wie ”monoton wachsend“ auf die Partialsummen, nicht auf die
	      Reihenglieder).
          
          Lösung:
          \begin{enumerate}
              \item 
          \end{enumerate}
	\item Zeige für \( 0 < x < 1  \) ist die \textit{geometrische Reihe}
	      \[ f_n := \sum_{k = 0}^{n} x^k \]

	      beschränkt und monoton, mithin konvergent. Was ist der Grenzwert der Reihe?
	      (Tipp: Die Summe kennen wir schon\dots )
          
          Lösung:
          \begin{enumerate}
              \item 
          \end{enumerate}
	\item Zeige, dass die harmonische Reihe mit Reihengliedern \( s_n := \frac{1}{n} \) keine obere
	      Schranke besitzt (die Partialsummen also beliebig groß werden).
          
          Lösung:
          \begin{enumerate}
              \item 
          \end{enumerate}
	\item Zeige, dass eine Folge höchstens einen Grenzwert besitzen kann!
	      Nimm dazu an, es gebe \( y_1 \neq y_2 \), die beide die Grenzwertbedingung erfüllen und
	      führe das zum Widerspruch durch Angabe eines \( \epsilon \), mit dem die weitere Bedingung
	      unmöglich erfüllt sein kann.
          
          Lösung:
          \begin{enumerate}
              \item 
          \end{enumerate}
	\item Ein Mann spaziert mit seinem Hund von seinem Haus zu einer Kneipe. Die
	      Entfernung zwischen Haus und Kneipe sei \( s \). Der Mann gehe dabei mit der
	      Geschwindigkeit \( v \). Dies ist dem Hund jedoch zu langweilig. Er läuft deswegen
	      doppelt so schnell zwischen der Kneipe und seinem Herrchen hin und her.
	      Das heißt, er startet am Haus zusammen mit seinem Herrchen, dreht um,
	      sobald er das Ziel erreicht, stoppt, wenn er wieder auf sein Herrchen trifft,
	      läuft dann wieder zur Kneipe,\dots

	      Berechne, welchen Weg der Hund zurücklegt, bis Herrchen und Hund gemeinsam die Kneipe erreichen.
	      Es gibt einen so genannten ”Mathematiker-Weg“ und einen so genannten
	      ”Physiker-Weg“. Versuche, beide zu finden.
          
          Lösung:
          \begin{enumerate}
              \item 
          \end{enumerate}
	\item  In Aufgabe \( 2.7 \) haben wir die Länge der Kochkurve zu berechnet, was uns auf
	      eine Folge \( (l_n) \) von Zahlen führt. Ebenso kann man die Fläche \( (a_n) \) unter der
	      Kochkurve berechnen, indem man die Flächen der Dreiecke aufaddiert, die
	      Schritt fur Schritt auf die Kurve ”draufgesetzt“ werden.
	      Sind diese Folgen jeweils beschränkt und/oder monoton?
          
          Lösung:
          \begin{enumerate}
              \item 
          \end{enumerate}
	\item Gib Folgen \( (a_n) \), \( (b_n) \) an mit \( \lim_{n \to \infty}(a_n) = \infty \),
	      \( \lim_{n \to \infty}(b_n) = \infty \) sowie:
	      \begin{enumerate}
		      \item \( \lim(a_n - b_n) = 0 \)
		      \item \( \lim(a_n - b_n) = + \infty \)
		      \item \( \lim( \frac{a_n}{b_n}) = 0 \)
	      \end{enumerate}
          
          Lösung:
          \begin{enumerate}
              \item 
          \end{enumerate}
	\item \( O \)- Notation: zeig, dass gilt:
	      \begin{enumerate}
		      \item \( 6n^4  \in O(3n^4) \)
		      \item \( 16n^3 \in O(2^n) \)
		      \item \( n^2 \notin O(n) \)
	      \end{enumerate}
	      Gib im Fall von "`\( \in \)"' ein entsprechendes \( n_0 \) an, so dass Bedingung
	      (8.1) erfüllt ist.
          
          Lösung:
          \begin{enumerate}
              \item 
          \end{enumerate}
	\item Bestmme alle Häufungspunkte und den größten Häufungspunkt von
	      \[ a_n = \sin \Big( \frac{\pi}{4}n \Big) \]
	      (Benutze dabei die Gleichheit \( \sin( \pi / 4 ) = \sqrt{2} / 2 \)).
          
          Lösung:
          \begin{enumerate}
              \item 
          \end{enumerate}
	\item Prüfe jeweils auf Konvergenz und bestimme ggf. den Grenzwert.

	      Hinweis: Wenn \( \lim_{n \to \infty} a_n = \alpha \) und
	      \( \lim_{n \to \infty} b_n = \beta \) dann ist
	      \( \lim_{n \to \infty} (a_n + b_n) = \alpha + \beta \).

	      Ebenfalls gilt: \( \lim_{n \to \infty} (a_n \cdot b_n) = \alpha \cdot \beta \) und
	      - wenn \( \beta \)  und alle \( b_n \neq 0 \) sind -
	      \( \lim_{n \to \infty }(a_n / b_n) = \alpha / \beta \)
	      \begin{enumerate}
		      \item \( a_n = \frac{ 3n +2(-1)^n }{ n } \)
		      \item \( b_n = \frac{ nx^n }{ nx^n + 1 } \)
		            \( x \in \mathbb{R}, x > 1 \)
	      \end{enumerate}
          
          Lösung:
          \begin{enumerate}
              \item 
          \end{enumerate}
\end{enumerate}
\end{document}