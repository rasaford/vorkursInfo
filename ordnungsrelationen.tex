\documentclass[main.tex]{subfiles}
    \usepackage{amsmath}
    \usepackage{amsfonts}
    \usepackage{listings}
    
\begin{document}
\begin{enumerate}
	\item Was für partielle Ordnungen und was für totale Ordnungen gibt es
	      auf zweielementigen Mengen \( \{x, y\} \)?

	      Lösung:
	      \begin{enumerate}
		      \item partielle Ordnung:

		            \( R_1 = \{ (x,x), (y,y) \}  \)

		            totale Ordnung:

		            \( R_2 = \{ (x, x), (y, y), (x, y) \}  \)

		            \( R_3 = \{ (x, x), (y, y), (y, x) \}  \)
	      \end{enumerate}
	\item Führe das im Beweis verwendete Sortierverfahren für die Menge \( A = \{ d, b, c, a, f, e \} \)
	      mit der alphabetischen Sortierung durch. Verwende als Pivotelement \( z \) immer
	      das vorderste Element (am Anfang also \( d \)) und lasse die Reihenfolge der
	      Elemente in \( A_1 \) und \( A_3 \) so, wie sie in \( A \) waren (also nicht beim Zerlegen aus
	      Versehen sortieren)!

	      Lösung:
	      \begin{enumerate}
		      \item \[ A = \{ d, b, c, a, f, e \} \]
		            \[ A_1 = \{b, c, a \}, d, A_2 = \{ f,e \} \]
		            \[ A_3 = \{ a \}, b,  A_4 = \{ c \}, d, A_5\{ e \}, f, A_6 = \{\} \]
		            \[ A_7 = \{a, b, c\}, d, A_8 = \{e,f\} \]
		            \[ A = \{a, b, c, d, e, f\}  \]
	      \end{enumerate}
	\item Beweise, dass die \( A_i \) aus dem Beweis in der Vorlesung paarweise disjunkt sind.

	      Lösung:
	      \begin{enumerate}
		      \item Die von dem Algorithmus erzeugen Teilmengen sind disjunkt.
		            Die Elemente von \( A \) können in eine von \( 3 \) Mengen sortiert werden.
		            \[ A_1 = \{ x \in A: xRz \land x \neq z \} \]
		            \[ A_2 = \{ z \} \]
		            \[ A_3 = \{ x \in A: zRx \land x \neq z \} \]
		            Falls \( x = z \) kann durch die Definition der Mengen \( A_1, A_3 \) nur in
		            die Menge \( A_2  \) sortiert werden.
		            Falls \( x \neq z \) kann es in \textit{entweder} \( A_1 \) \textit{oder} \( A_2 \)
		            sortiert werden, da auf den Elementen von \( A \) eine partielle Ordnung existiert und
		            diese die Antisymmetrie fordert. Dadurch kann entweder \( xRy \) oder \( yRx \), aber \textit{nicht}
		            \( xRy \land yRx \) existieren. Somit wird \( x \) in eine der beiden Mengen sortiert.

		            Da \( A \) ebenfalls transitiv ist muss eine der beiden direkten Relationen zu \( x \)
		            existieren.
	      \end{enumerate}
	\item Eine schöne Darstellung einer Relation \( m \) auf einer endlichen (und nicht zu
	      großen) Menge \( A \) ist mittels einer Tabelle, bei der Zeilen und Spalten mit den
	      Elementen von \( A \) beschriftet sind (beides Mal dieselbe Reihenfolge wählen)
	      und in der wir genau dann in Zeile \( x \) und Spalte \( y \) ein Kreuz machen, wenn
	      \( xRy \) gilt.
	      \begin{itemize}
		      \item  Wie sieht die Tabelle für die Teilmengenrelation auf \( \mathcal{P}(\{ 1, 2 \}) \) aus?
		      \item Wie sieht man einer solchen Tabelle an, ob die Relation
		            \begin{itemize}
			            \item reflexiv
			            \item total und / oder
			            \item antisymmetrisch
		            \end{itemize}
		            ist?
		      \item Wie sieht die Tabelle einer Ordnung aus, wenn Zeilen- und Spaltenbeschriftungen
		            nach dieser Ordnung sortiert sind?
	      \end{itemize}

	      Lösung:
	      \begin{enumerate}
		      \item
		            \[
			            \begin{array}{ c|c|c|c|c }
				            x \subseteq y & \{\}   & \{ 1 \} & \{ 2 \} & \{ 1,2 \} \\
				            \hline
				            \{\}          & \times &         &         &           \\
				            \hline
				            \{1\}         & \times & \times  &         &           \\
				            \hline
				            \{2\}         & \times &         & \times  &           \\
				            \hline
				            \{1,2\}       & \times & \times  & \times  & \times    \\
			            \end{array}
		            \]
		      \item
		            \begin{itemize}
			            \item Reflexivität:

			                  Die Diagonale der Tabelle ist markiert.
			            \item Totalität:

			                  Wenn die Diagonale als Spiegelache betrachtet wird, muss auf mindestens
			                  einder der beiden Seiten eins der beiden Felder markiert sein
			            \item Antisymmetrie:

			                  Bei der an der Diagonalen gespiegelten Tabelle ist höchstens eins
			                  der beiden Felder markiert.
		            \end{itemize}
		      \item
		            Ordnung: \( \leq \) auf den Natürlichen Zahlen
		            \[
			            \begin{array}{ c|c|c|c|c }
				            x \leq y & 1      & 2      & 3      & 4      \\
				            \hline
				            1        & \times &        &        &        \\
				            \hline
				            2        & \times & \times &        &        \\
				            \hline
				            3        & \times & \times & \times &        \\
				            \hline
				            4        & \times & \times & \times & \times \\
			            \end{array}
		            \]
		            Es sind alle Felder unter der Diagonalen markiert, da die Ordnung reflexiv, total (transitiv) und
		            antisymmetrisch ist.
	      \end{enumerate}
	\item  Für eine endliche Menge \( A \) sei jedes Element \( x \in A \) mit einer Rangziffer
	      \( r(x) \in \mathbb{N} \) versehen. Wir betrachten die Relation \( R \) auf \( A \) mit
	      \[ xRy: \Leftrightarrow r(x) \leq r(y) \]
	      \begin{itemize}
		      \item Zeige, dass \( R \) transitiv, reflexiv und total ist.
		      \item Unter welcher Bedingung an die Beschriftungen \( r(x) \) ist \( R \) antisymmetrisch,
		            also eine Ordnung?
	      \end{itemize}

	      Lösung:
	      \begin{enumerate}
		      \item Da \( r(x) \in \mathbb{N} \) und die Relation \( \leq \) auf \( \mathbb{N} \)
		            eine totale Ordnung bildet ist \( r(x) \leq r(y) \) transitiv, reflexiv und total.

		            Da \( R  \) durch diese definiert wird, hat \( R \) auch diese Eingenschaften.

		      \item Wenn alle Beschriftungen für die verschiedenen Elemente von \( A \) unterschiedlich
		            sind, ist \( R \) antisymmetrisch.
		            \[ \forall x,y \in A: r(x) \neq r(y) \]
		            Dann gilt für jedes \( x,y \in A \) \textit{entweder} \( xRy \)
		            \textit{oder} \( yRx \).
		            \[ \forall x,y \in \mathbb{N}, x \neq y: x \leq y \oplus x \leq y \]
	      \end{enumerate}
	\item Wenn man für eine endliche Menge \( A \) mit \( n = |A| \) Elementen nur eine
	      partielle Ordnung hat, funktioniert das Sortieren nicht. (Warum nicht?)

	      Es gibt aber einen entsprechenden Satz, der besagt, dass man \( A \) mit seiner
	      partiellen Ordnung topologisch sortieren kann, d.h. es gibt immer (mindestens)
	      eine Numerierung \( A = \{ x_1, x_2, \dots, x_n \} \), für die gilt:
	      \[ \forall i, j \in \{ 1, \dots, n\} : x_iRx_j \Rightarrow r \leq j \]

	      Beweise mithilfe dieses Satzes folgende Aussage: Zu jeder partiellen Ordnung
	      \( R \) auf einer endlichen Menge \( A \) gibt es eine totale Ordnung \( R' \)
	      mit \( R \subseteq R' \). (In Worten:
	      Jede partielle Ordnung auf einer endlichen Menge kann ich ”durch
	      Hinzufugen von Pfeilen“ zu einer totalen Ordnung ausbauen.)

	      Lösung:
	      \begin{enumerate}
		      \item Es wird \( A \) mit \( R \) topologisch sortiert. Dadurch erhält man
		            \[ A = \{x_1, x_2, \dots, x_n\} \]
		            Wenn man nun jedem \( x \in A \) eine Rangziffer, die dem Index in
		            \( A \) entspricht zuweist entsteht somit nach der vorherigen Aufgabe
		            eine totale Ordnung auf diesen. (\( r(x_i) \in \mathbb{N} \))
		            \[ r(x_1) = 1, r(x_2) = 2, \dots, r(x_n) = n \]
		            Nun ist noch zu zeigen, dass \( R \subseteq R' \)

		            \( xRy \Rightarrow r(x) \leq r(y) \stackrel{\text{def. von } R'}{\Rightarrow} xR'y \)

		            Dies ist der Fall, da \( R' \) als totale Ordnung definiert ist:
		            \[ \forall i,j \in \{ 1 \dots n\} : i \leq j \Leftrightarrow x_iRx_j \]
	      \end{enumerate}
\end{enumerate}
\end{document}