\documentclass[main.tex]{subfiles}
    \usepackage{amsmath}
    \usepackage{listings}
    
\begin{document}
\begin{enumerate}
	\item Man zeige direkt anhand der \( \epsilon \)-\( \delta \)-Definition die Stetigkeit der Funktion
	      \( f(x) = |x| \).
	      Wie kann man anhand der \( \epsilon \)-\(\delta \) Definition zeigen, dass die Signumsfunktion
	      \[ sign(x)= \begin{cases}
			      1  & x > 0, \\
			      0  & x = 0, \\
			      -1 & x < 0,
		      \end{cases}
	      \]
		  in \( x = 0 \) nicht stetig ist? 
		  
		  Lösung:
		  \begin{enumerate}
			  \item 
		  \end{enumerate}
	\item Man zeige
	      \[ x = e^{ln(x)} \]
	      und leite durch beidseitiges Differenzieren eine Regel für
	      die Ableitung des Logarithmus her.
		  
		  Lösung:
		  \begin{enumerate}
			  \item 
		  \end{enumerate}
	\item Bestimme die Ableitung der Funktion
	      \[ f(x) = -x + x \ln(x) \]
	      Was lässt sich daraus mithilfe der Gleichung
	      \( \int f'(x) dx = F(x) + C \) folgern?
		  
		  Lösung:
		  \begin{enumerate}
			  \item 
		  \end{enumerate}
	\item Man leite mit Hilfe der Kettenregel die Ableitung von \( \frac{1}{ g(x) } \) und anschließend
	      mit der Produktregel die Ableitung von \( \frac{ f(x) }{ g(x) } \) her.
		  
		  Lösung:
		  \begin{enumerate}
			  \item 
		  \end{enumerate}
	\item In der Vorlesung wurde die Ableitungsregel
	      \[ \frac{d}{dx} x^n = nx^{n-1} \]
	      direkt anhand der Definition der Ableitung gezeigt. Beweise diese Ableitungsregel
	      noch einmal mit vollständiger Induktion.
		  
		  Lösung:
		  \begin{enumerate}
			  \item 
		  \end{enumerate}
	\item Der Mittelwertsatz der Differentialrechnung lautet:

	      Die Funktion \( f \) sei im Intervall \( [a,b] \) stetig differenzierbar.
	      Dann existiert ein \( \xi \) mit
	      \[ f(b) - f(a)  = f'( \xi )( b - a ) \]
	      \begin{enumerate}
		      \item Was bedeutet der Satz anschaulich?
		      \item Beweise den Satz von Rolle:

		            Die Funktion \( f \) sei im Intervall \( [a,b] \) stetig differenzierbar und
		            es gelte \( f(a) = f(b) \). Dann besitzt der Graph von \( f \) zwischen
		            \( a \) und \( b \) mindestens einen Punkt mit waagrechter Tangente.
	      \end{enumerate}
		  
		  Lösung:
		  \begin{enumerate}
			  \item 
		  \end{enumerate}
	\item Wir betrachen die Funktion
	      \[ f(x) = \frac{ 3^3 + x^2 - 4 }{
			      4x^2 - 16
		      } \]
	      \begin{enumerate}
		      \item Man gebe den maximalen Definitionsbereich von \( f \) an.
		      \item Zeige die Identität
		            \[ f(x) - \frac{1}{4} = - \Bigg( f(-x) - \frac{1}{4} \Bigg)  \]

		            Was lässt sich daraus hinsichtlich der Symmetrie des Graphen von
		            \( f \) folgern?
		      \item Berechne die Schnittpunkte des Graphen von \( f \)  mit den Koordinatenachsen.
		      \item Bestimme alle Asymptoten von \( f \) und berechne die Schnittpunkte des
		            Graphen von \( f \)  mit der schiefen Asymptote.
		      \item Berechne die ersten beiden Ableitungen von \( f \).
		            Kontrolle:
		            \[ f'(x) = \frac{3}{4} \cdot x^2 \cdot \frac{ x^2 - 12 }{ (x^2 - 4)^2 } \]
		      \item Bestimme alle Extrempunkte.
		      \item Untersuche \( f \) auf Wendepunkte.
		      \item Zeichne den Graphen von \( f \) unter der Verwendung aller bisherigen Resultate.
	      \end{enumerate}
		  
		  Lösung:
		  \begin{enumerate}
			  \item 
		  \end{enumerate}
\end{enumerate}
\end{document}