\documentclass[main.tex]{subfiles}
    \usepackage{amsmath}
    \usepackage{listings} 
    \usepackage{amsfonts}
    \usepackage{multicol}

\begin{document}
\begin{enumerate}
	\item Gegeben sei das lineare Gleichungsystem
	      \[ \begin{array}{cccc}
			      -x_1 & +2 x_2 & = & 2 \\
			      2x_1 & -x_2   & = & 2 \\
		      \end{array} \]
	      \begin{enumerate}
		      \item Löse das System zunächst graphisch.
		      \item Eliminiere nun mittels der ersten Gleichung das \( x_1 \)
		            in der zweiten Gleichung
		      \item Löse das so geänderte System noch einmal graphisch.
		      \item Berechne schließlich aus dem geänderten System die Lösung.
	      \end{enumerate}

	      Lösung:
	      \begin{enumerate}
		      \item
		      \item \[ \begin{pmatrix}
				            -1 & 2 & 2 \\
				            0  & 1 & 2 \\
			            \end{pmatrix} \]
		      \item
		      \item  \[
			            \begin{pmatrix}
				            -1 & 2  & 2 \\
				            2  & -1 & 2 \\
			            \end{pmatrix}
			            \rightsquigarrow
			            \begin{pmatrix}
				            1 & 0 & 2 \\
				            0 & 1 & 2 \\
			            \end{pmatrix}
		            \]
	      \end{enumerate}
	\item Löse das lineare Gleichungsystem
	      \[ \begin{array}{cccccc}
			      2x_1  & +2x_2 & -x_3  & -2x_4 & = & -1 \\
			      4x_1  & +4x_2 & -3x_3 & -x_4  & = & 5  \\
			            & 3x_2  & +x_3  & +x_4  & = & 1  \\
			      -2x_1 & +4x_2 & +4x_3 & +2x_4 & = & -2 \\
		      \end{array} \]

	      Lösung:
	      \begin{enumerate}
		      \item \[
			            \begin{pmatrix}
				            2  & 2 & -1 & -2 & -1 \\
				            4  & 4 & -3 & -1 & 5  \\
				            0  & 3 & 1  & 1  & 1  \\
				            -2 & 4 & 4  & 2  & -2 \\
			            \end{pmatrix}
			            \rightsquigarrow
			            \begin{pmatrix}
				            1 & 0 & 0 & 0 & 1  \\
				            0 & 1 & 0 & 0 & 0  \\
				            0 & 0 & 1 & 0 & -1 \\
				            0 & 0 & 0 & 1 & 2  \\
			            \end{pmatrix}
		            \]
	      \end{enumerate}
	\item Das Lösen eines LGS nach dieser Methode benötigt bei \( n \) Unbekannten etwa
	      \( n^3/3 \) Operationen (Additionen und Multiplikationen). Angenommen, unser
	      Rechner schafft \( 100 \) Millionen Operationen pro Sekunde — wie lange braucht
	      er dann fur ein LGS mit \( 10 \), mit \( 1000 \), mit \( 100000 \) Unbekannten?

	      Lösung:
	      \begin{enumerate}
		      \item Allgemeine Formel für \( n \) Unbekannte
		            \[ t(n) := \frac{n^3}{3} \cdot \frac{1}{100 \cdot 10^6} s
			            = \frac{n^3 }{3 \cdot 10^{8}}  s\]
		            \[ \begin{array}{ccccc}
				              & 10         & 100     & 1 000    & 100 000            \\
				            \hline
				            t & 3.33 \mu s & 3.33 ms & 3.33 sec & 38.58 \text{ Tage} \\
			            \end{array} \]
	      \end{enumerate}
	\item Für eine Matrix \( A \in \mathbb{R}^{ r \times s } \) (d.h. \( r \) Zeilen und \( s \)
	      Spalten, Koeffizienten aus
	      \( \mathbb{R} \)) und einen Vektor \( b \in \mathbb{R}^s \) ist das Matrix-Vektor-Produkt
	      \( c = A \cdot b \in \mathbb{R}^r \) definiert, bei dem in Zeile \( i \) das Skalarprodukt
	      aus der Zeile \( i \) von \( A \) und dem Vektor \( b \) gebildet wird:
	      \[ c_i = \sum_{ k = 1 }^{s} a_{i,k} \cdot b_k \]

	      Berechne folgendes Matrix-Vektor-Produkt
	      \[  \begin{pmatrix}
			      2  & 2 & -1 & -2 \\
			      4  & 4 & -3 & -1 \\
			      0  & 3 & 1  & 1  \\
			      -2 & 4 & 4  & 2  \\
		      \end{pmatrix}
		      \cdot \begin{pmatrix}
			      1  \\
			      0  \\
			      -1 \\
			      2  \\
		      \end{pmatrix}  \]
	      und überprüfe die Ergebnisse aus der Aufgabe 11.2

	      Lösung:
	      \begin{enumerate}
		      \item \[
			            \begin{pmatrix}
				            2  & 2 & -1 & -2 \\
				            4  & 4 & -3 & -1 \\
				            0  & 3 & 1  & 1  \\
				            -2 & 4 & 4  & 2  \\
			            \end{pmatrix}
			            \cdot \begin{pmatrix}
				            1  \\
				            0  \\
				            -1 \\
				            2  \\
			            \end{pmatrix}
			            = \begin{pmatrix}
				            2 + 1 -4   \\
				            4 + 3 -2   \\
				            -1 + 2     \\
				            -2 + 4 + 4 \\
			            \end{pmatrix}
			            = \begin{pmatrix}
				            -1 \\
				            5  \\
				            1  \\
				            -2 \\
			            \end{pmatrix}
		            \]
	      \end{enumerate}
	\item für welche Werte von \( a \) ist folgendes LGS lösbar?
	      Was sind dann die Lösungen?
	      \[ \begin{array}{ccccc}
			      x_1   & +x_2  & +x_3  & = & 2 \\
			      x_1   & +4x_2 & +3x_3 & = & 4 \\
			      -2x_1 & -3x_2 & -x_3  & = & a \\
		      \end{array} \]

	      Lösung:
	      \begin{enumerate}
		      \item \[
			            \begin{pmatrix}
				            1  & 2  & 1  & 2 \\
				            1  & 4  & 3  & 4 \\
				            -2 & -3 & -1 & a \\
			            \end{pmatrix}
			            \rightsquigarrow
			            \begin{pmatrix}
				            1 & 1 & 0 & -a-2 \\
				            0 & 1 & 1 & 1    \\
				            0 & 0 & 0 & a+3  \\
			            \end{pmatrix}
		            \]
		            \begin{multicols}{3}
			            \begin{itemize}
						\item[] \( 0 + 0 + 0 = a + 3 \)
						\( a = -3 \)

				            \item[] \( x_2 + x_3 = 1 | -x_3 \)

				                  \( x_2 = 1 - x_3 \)
				            \item[] \( x_1 + x_2 = 1 \)

				                  \( x_1 + (1-x_3) = 1 \)

				                  \( x_1 = x_3 \)
			            \end{itemize}
		            \end{multicols}
	      \end{enumerate}
\end{enumerate}
\end{document}