\documentclass[a4paper, 10pt]{article}
    \usepackage[subpreambles=true]{standalone}
    \usepackage[ngerman]{babel}
    \usepackage[utf8]{inputenc}
    \usepackage[T1]{fontenc}
    \usepackage{hyphenat}
    \hyphenation{Mathe-matik wieder-gewinnen}
    \usepackage{amsmath}
    \usepackage{import}
    \usepackage[landscape, twocolumn, margin=2cm ]{geometry}

    \title{Mathematik Vorkurs für Informatiker 2017 Lösungen}

\begin{document}
\maketitle

\section{Einstimmung}
\import{sections/}{einstimmung}

\clearpage
\pagebreak
\section{Potenzen und Polynome}
\import{sections/}{polynome}

\clearpage
\pagebreak
\section{Logarithmen}
\import{sections/}{logarithmen}

\clearpage
\pagebreak
\section{Aussagenlogik und Beweise}
\import{sections/}{logik}

\clearpage
\pagebreak
\section{Mengen}
\import{sections/}{mengen}

\clearpage
\pagebreak
\section{Relationen}
\import{sections/}{relationen}

\clearpage
\pagebreak
\section{Ordnungsrelationen}
\import{sections/}{ordnungsrelationen}

\clearpage
\pagebreak
\section{Abbildungen}
\import{sections/}{abbildungen}

\clearpage
\pagebreak
\section{Folgen}
\import{sections/}{folgen}

\clearpage
\pagebreak
\section{Analysis}
\import{sections/}{analysis}

\clearpage
\pagebreak
\section{Kombinatorik}
\import{sections/}{kombinatorik}

\clearpage
\pagebreak
\section{Lineare Gleichungssysteme}
\import{sections/}{lgs}

\end{document}